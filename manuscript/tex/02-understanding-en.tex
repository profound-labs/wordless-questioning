We break knowlege into parts and we speak about practice in gradual
steps, but understanding happens at once. What is going to be valuable
is not the accumulation of facts, but the letting go, the release from
stress and suffering.

Accumulating experiences and successes our peace depends on more and
more external things, and we never arrive at the place where we can
stop. This wandering around is exhausting, that way we are full of
doubts and feelings of lack. We stop and understand, that it is not more
things which we need, but the letting go of that need.

When we keep counting what we have achieved and where we are compared to
others, our face is more and more bitter -- even if we are successful,
if we cling to that and identify with that success, we are strengthening
the external dependency and internal feeling of lack. The advertisements
certainly would like it very much that we believe in this, to always see
ourselves in the next thing.

The worldly values and rewards are always going to be fundamentally
empty. But we can notice that we are able to stop. The little is enough,
we don't require much, an in this restraint we find a strength and
energy which is always at hand, and not threatened by external factors.

We watch the breathing, and how the senses operate. The eye sees forms,
and a sensation appears. The ear hears sounds, the body feels solidity,
hot and cold. With contact a feeling appears. If there is contact
between the sense and the sense-object, the process doesn't depend on
us, we don't have a choice whether the sensation should appear. When the
contact between the sense and sense-object is broken, the sensation
ceases, which was our experience, and we can't control its cessation,
either.

When we don't notice this, we think that the feeling will be ours, we
are going to get something from it, and so we cling to it. The need and
anxiety is made possible by this incorrect understanding, and when we
are disappointed in our expectations, we don't investigate our attitude,
but instead we think that another, new thing is not going to behave like
this.

Observing sense-contact this way, the feelings are no longer attractive
or repulsive, because we see that neither is going to be stable and
reliable. This is not numbness -- in meditation we turn toward
experience, not away from it -- we remain alert and sensitive to the
experience, but it no longer controls us, and we remain calm.

We don't believe in the stories about who we are and what we should
feel. We are with ourselves, the feeling of lack is replaced with
satisfied contentment, even though we didn't have to get anything for
it.

This is the way we learn about the teaching of the Buddha. It helps to
think about in parts, and both our knowlege and our questions can be
expressed more clearly this way, but in the meantime we remember that it
is not this knowlege which is going to be valueable, but that we can
turn this toward understanding our experience.

We look at our experience as part of a process. First we notice the
surface experience -- especially when it is painful, for pleasant
experiences somehow we don't require so much explanation.

We seek the roots of the processes, we investigate events as causes and
results. Our experience is what we feel as a result. Our attitude to
this determines how we relate to it and how we act, this becomes another
cause, the results of which -- be it pleasant or painful -- we are going
to experience.

The fundamental emphasis is on our attitude, on our intention and
actions. This is where we have free choice, and ability to influence
what happens to us. We can't control the world around us, but our
attitude opens or closes the doors to the actions we see as possible,
and our actions create the situation, where we live our lives.

The starting position is to acknowlede that the stress or suffering is
here. As information we can easily accept this, that yes, there is
stress and suffering in the world, but when we ourselves experience it,
we like to rather pay attention to something else, or tend to rather
blame somebody else for it, anything but save us from having to
acknowledge and deal with it.

The instruction here is that the only way forward is to turn toward the
suffering, and investigating it, we seek a way of understanding. This is
the First Noble Truth in the teaching of the Buddha -- there is
suffering, and the attitude which is going to ennoble will be if we turn
toward it and understand it.

What do we understand? That the suffering didn't come from nothing, but
that it is the result of earlier causes, the we are experiencing the
inevitable results of these. This is a great step forward, because if we
can examine our situation in this way, we are no longer helpless -- and
even though we don't understand every little factor, it is already a
relief that perhaps we are able to change something.

The Second Noble Truth points out that we find the cause of suffering in
ourselves -- our wish that things were otherwise than their nature
dictates, our tight clinging to what is impermanent, breaks up, and not
possible to keep. The instruction, the noble attitude here is that we
must let go of this thirsty craving and clinging. Our experinces, the
transitory things in our lives are not worth clinging to, this always
leads to suffering, because their nature is impermanence. When we see,
that in reality there is nothing in them which we can keep, we
understand how we caused ourselves pain by clinging to them.

With the cessation of the factor which caused it, the result, the
suffering ceases as well. The Third Noble Truth directs our attention
toward this -- there is a solution, we are not obliged to live in
bitterness and meaningless struggle. The advice, the noble attitude is
that this connection should be realized for ourselves, and with letting
go, allowing the suffering to cease.

Even if can't let go right away, it is already a relief to see that the
connection is true -- 'if I could let go, I wouldn't suffer from it'.
This is already half of the work, until now we were wandering without a
map, but now there is a way forward.

The Fourth Noble Truth describes the practice of the way. The Buddha
divided it into eight factors, which incorporate the situations of daily
life and the development of meditation as well.

The parts of the Eightford Path are (1) understanding, (2) intention,
(3) speech, (4) action, (5) livelihood, (6) mindfulness and (8)
concentration. When they are aligned with the truth, we call then
\emph{right}: Right Understanding, Right Intention and so on. Breaking
it down to parts helps the investigation, it is easier for us to think
this way, but the factors strengthen and support each other, and the
practice is realized as an integrated whole.

In the turbulent situations of life we can't stop to count factors, a
useful tool will have to be portable and easily accessible. When we read
and ponder the meaning, we have time to turn the words this way and
that, but the mindful attention, as an abstract idea doesn't help much
-- it is only valueable when it is at hand in the present moment.
