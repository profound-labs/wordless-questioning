\documentclass{article}
\usepackage{dia-page}
\geometry{paperwidth=105mm, paperheight=180mm, vmargin=1pt, hmargin=1pt, nohead, nofoot}

\begin{document}
\diaSmall
\centering

\includegraphics[width=60mm]{illusion-cube-rotated.jpg}

\vspace*{30pt}

Látsz egy kockát a fenti és alábbi diagramon?\\
Van ott egy kocka rajzolva?

\vspace*{30pt}

\begin{tikzpicture}[font=\diaSmall, node distance=2mm]

  \node {\parbox{70mm}{\centering \includegraphics[width=60mm]{illusion-cube.jpg}}};

  \node (c) [circle, minimum width=75mm] {};

  \draw (0*360/6+90: 4cm) node
  {\parbox{30mm}{\centering Szem\\ formákat lát}};

  \draw (1*360/6+90: 4cm) node
  {\parbox{20mm}{\centering Elme mentális tárgyakat kognizál}};

  \draw (2*360/6+90: 4cm) node
  {\parbox{15mm}{\centering Test\\ tárgyakat\\ tapint}};

  \draw (3*360/6+90: 4cm) node
  {\parbox{30mm}{\centering Nyelv\\ ízeket ízlel}};

  \draw (4*360/6+90: 4cm) node
  {\parbox{20mm}{\centering Orr\\ szagokat\\ szagol}};

  \draw (5*360/6+90: 4cm) node
  {\parbox{20mm}{\centering Fül\\ hangokat\\ hall}};

  \node [above=23mm and 0mm of c.center, font=\diaSmall] {Én?};

\end{tikzpicture}%

\end{document}
