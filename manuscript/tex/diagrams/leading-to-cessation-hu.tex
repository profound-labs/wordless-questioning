\documentclass{article}
\usepackage{dia-page}
\geometry{paperwidth=120mm, paperheight=180mm, vmargin=1pt, hmargin=1pt, nohead, nofoot}

\begin{document}

\centering

\begin{tikzpicture}[font=\diaSmall, node distance=2mm]
  \node (realm) [tight ring]
  {\parbox{20mm}{\centering Létezés az érzékek világában}};

  \node (wise) [tight box, below=22mm of realm]
  {\parbox{38mm}{\centering alapos, bölcs vizsgálódás\\ \emph{yoniso manasikāra}}};

  \node (restraint) [tight box, below=22mm of wise]
  {\parbox{38mm}{\centering érzéki visszafogottság,\\ moralitás és erény irányítják az életcélokat és tetteket}};

  \node (control) [below left=5mm and 4mm of restraint.south, anchor=north east]
  {\parbox{38mm}{\centering\diaTiny
      Irányítást nyer a mentális élet felett,
      a tudatosság fenntartja a kiegyensúlyozottságot}};

  \node (taints) [tight box, below=23mm of restraint]
  {\parbox{38mm}{\centering kényszeres hajlamok (\emph{āsava}) gyengülnek}};

  \node (cessation) [tight ring, below=21mm of taints]
  {\parbox{20mm}{\centering A szenvedés megszűnéséhez vezet}};

  \node (group) [rectangle, left=3mm of wise, minimum width=67mm, minimum height=15mm, draw] {};

  \node (fetters) [tight box, below right=2mm and 2mm of group.north west, minimum height=2\baselineskip]
  {\parbox{22mm}{\centering Belátás elhagyja a Béklyókat}};

  \node (hindrances) [tight box, below left=2mm and 2mm of group.north east, anchor=north east, minimum height=2\baselineskip]
  {\parbox{31mm}{\centering Helyes Erőfeszítés elhagyja az Akadályokat}};

  \node [above left=15mm and 3mm of fetters.north west, anchor=south west]
  {\parbox{72mm}{\raggedright\diaTiny\setlength{\parskip}{5pt}
      Három kényszeres hajlam (\emph{āsava}):\\
      \emph{kāmāsava:} vágy az érzéki örömökre\\
      \emph{bhavāsava:} vágy a létezés állapotaira\\
      \emph{avijjāsava:} tudatlanság az érzéki tapasztalat valódi természetére
    }};

  \node (thinking) [below=18mm of group]
  {\parbox{40mm}{\centering\diaTiny
      Így gondolkodik:\\
      Van szenvedés. Ennek van okozati eredete,
      megszűnése, és a megszűnéséhez vezető út.
    }};

  \node (noble) [tight ring, below=17.9mm of thinking]
  {\parbox{20mm}{\centering Látja\\ a Négy Nemes Igazságot}};

  \node (reinforces) [below left=7mm and -5mm of noble]
  {\parbox{24mm}{\centering\diaTiny
      elengedi a ragaszkodást és személyes azonosulást
    }};

  \node (seeing) [below=20mm of noble]
  {\parbox{35mm}{\centering\diaTiny
      Az érzéki tapasztalatot folyamatként látja:\\
      állandótlan, elégtelen, éntelen
    }};

  \draw [smallish arrow] (realm) to (wise);
  \draw [smallish arrow] (wise) to (restraint);
  \draw [smallish arrow] (restraint) to (taints);
  \draw [smallish arrow] (taints) to (cessation);

  \draw [smallish arrow] (taints) to (seeing);
  \draw [smallish arrow] (seeing) to (noble);
  \draw [smallish arrow] (noble) to (taints);

  \draw [line] (group) to (thinking);
  \draw [smallish arrow] (thinking) to (noble);

  \draw [smallish arrow, bend left=40] (group.north) to (wise.north);
  \draw [smallish arrow, bend left=40] (wise.south) to (group.south);

\end{tikzpicture}%

\end{document}
