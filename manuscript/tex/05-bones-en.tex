\hypertarget{bones-1}{%
\chapter{Bones}\label{bones-1}}

Meditation can be practiced in the standing posture as well. After the
sitting posture, it allows the joints and the muscles of the body to
relax. It requires more attention to maintain balance, and so the
supporting structures of the body are more easily felt, such as the
solidity of the bones.

The bones are fascinating because they make up the core, the rigid parts
of the body which determine its shape, what it can and cannot do. We
take our image in the mirror for granted, and have a relatively constant
image of ourselves, but if the bones were different, the body would be a
different shape, and it would change not only our appearance, but also
how we live.

While standing, take some time to find a balanced posture. Sway the body
left and right a bit and feel the center of weight. Develop that sense
that you are holding the body upright, so that it doesn't fall. Relax
the knees and let them bend a bit. Don't lock the joints straight, it
stresses them and your posture becomes rigid.

Allow yourself flexibility, small adjustments to the posture as the
muscles get used to the situation. Feel out the balance in standing and
watch the body which you are holding this way. Gravity is pulling it
down, there is a pressure on the ground. Collect the hands in front of
the abdomen, one palm on the other, in a comfortable manner.

Take a deep breath, and watch how the posture changes with the
breathing. The ribs in the chest open outwards, the shoulder rises, the
center of weight shifts slightly. Is there something limiting the
breathing? Take care to stand upright, so that the shoulders are not
hunched, which blocks the open breathing. Watch the balance of the head,
find the position where the head sits on top of the spine by its own
weight, not leaning forward or pulled backward. It is best if the gaze
is not looking directly ahead, but rather slightly down in front of us,
because the coming and going will distract our attention. Pulling in the
chin gently, don't push it out, the gaze is directed a few meters in
front of us on the floor, the crown of the head rises up a bit, as if
pushing the sky.

The position of the head controls the posture of the upper body to a
great degree. Because of sitting on chairs, we fall into a habit of
carrying the head in a manner pushed forward. This creates a tension in
the back muscles along the spine, but we can't control these muscles
consciously. Experiment, what does it feel like, to pull the head back a
bit. The muscles in the back relax.

The balanced posture stants like carefully positioned stones on top of
each other. The bones sit on one another, like stones. Gravity is enough
to hold it, it is balanced without force, and gives us a light, pleasant
feeling.

Feel how the bones connect. There is this perception of an inner
structure which supports the body from the inside. Rigid pieces, stacked
on top of each other. There is a sensation in the legs, a rigid
perception, which are the long leg bones. There is pressure. The hip
bone is resting on top of the legs, the torso moves joined on to it. The
rib-cage expands and contracts with the breathing. The spine is holding
the weight in a curve. The head is sitting on top of the spine. The
skull bones are stretching the skin on the face.

It is made out of pieces. Pieces that in some places hard and rigid, and
some places soft and flexible, these give it shape. There is hair on top
of the head. Some hair on the hands and other parts.

The habitual perception of the body, of our body and of other people's
body, is that we see it as one unit, as one thing. From that perspective
develops an obsession that it has some ideal way to be. That it has to
be a certain shape, height, different qualities.

These are worldly judgements, perceptions which our society drill into
us. Advertisements and media messages reinforce these expectations and
we believe in them, but when we look closer we see this is a twisted
perception, not according to reality.

These social expectations create an anxiety about our appearance, we are
concerned about what other people think about our body, and we create
opinions about their bodies.

We can watch how this anxiety is conditioned when parts of the body are
separated. We can be concerned about our hair for example, but only when
it is on our head. When it is cut off, we are not anxious about a pile
of hair on the floor. Similarly, when cutting our nails, what is that
point when it is no longer 'me' and 'mine'?

When we contemplate it this way, as something made up of pieces, we can
see the body is not one unit, it is made of pieces and parts which have
their own nature, and behave according to that, not concerned with our
opinions or others. Bones, skin, hair, teeth and nails, they are the way
they are. Contemplating this gives us confidence in accepting them with
kindness.

The body is a blessing. This meditation is not to develop aversion or
negative emotions toward the body. Health is a blessing, it supports us
in everything we do. The Buddha called health the greatest treasure.

\begin{verbatim}
USE HU TEXT
\end{verbatim}

Breathing in, breathing out, attention is turned inwards, noticing
'there is a body'. The mind is not seeking anything, not going out
somewhere. From the feet, to the legs, abdomen, chest, arms, the skull
and the head, you can feel the skin as it is stretched against the bone
of the skull.

Watching the body like this is like watching the rain, there is nothing
to do, nothing to decide, and rain goes on without us having to get
involved.

Awareness of the body stops the anger, resentment and inner blaming.
There is no escape. Continuing the thinking we only dig ourselves
deeper, but we can't see that at the time. Somehow we feel it is
important to be angry, although only the time passes painfully with it.
We wish for it to be over finally, and let things continue smoothly.

We observe the parts of the body and see that they don't carry a story
with them. We can breathe a sigh of relief that we are not chained to
the stories, we are creating them.

We can always return to this attention, one in-breath and out-breath is
enough to remember arising and ceasing, and our problems become like
stories in an old newspaper. We get tired of untangling the threads, as
if we had to interpret someone else's dreams. What is real, is always
here in our present experience. What becomes important is not what the
story is, but whether we can give our attention to where we are now.

Awareness of the body loosens the desires and leads us to recognize that
we are fortunate to be here.

Where do we want to get to? We can start now. If it is a truly
worthwhile thing, it is almost certainly difficult as well. If it is
difficult, it is almost certain we don't know what to do. Uncertainty
has to be part of the plan.

However, is it probably not complicated to start. We can ask ourselves,
if we had a time-machine, would we go back a few years, so that we can
start then already? If we had already started a few years ago, we would
be glad that by now, at least we have some information about the
situation. In the present, we can do this favour for our future self. We
can start now, and imagine, that in a few year's time we look back and
thank ourselves, to have started to clear the fog.

When we don't set a clear intention, we are just drifting, and \emph{we
don't particularly mind} being here, but the mind is grey with no life,
almost trying to hide itself and be invisible. We do end up being grey
and invisible like that. There is nothing wrong happening, but there
isn't any brightness in being there.

Looking back on the present with the eyes of our future self, would we
come back, because \emph{we want} to be here? We can be cynical and
think of the worst, but surprisingly often the answer doesn't start
analysing the situation, but instead, like when we travel to a new
place, we are grateful that we are fortunate to be here where we are
now. There are things to do later, but we can already say 'thank you'
for what we could experience until now.

We don't stop often enough to notice when we are happy and peaceful.
When the mind is clear and calm, it is naturally grateful for what is
here, it is able to say 'thank you' for the blessings we received in our
life. The present is good, and whatever way it develops for the rest of
our life, we are able say 'thank you'.

We are not creating something, with a clear intention we recognize what
is here. It is not a matter of strength or ability, these are bound to
time and circumstance. The resolution, the recognizing attention turned
inwards, is not bound to a give circumstance. The result is right
perspective, in which we can see the right place of things, and what to
do with them -- or to just stop, give attention and breathe.
