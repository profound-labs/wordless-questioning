\hypertarget{boat-1}{%
\chapter{Boat}\label{boat-1}}

At the beginning of the meditation we calm the body and quiet the
thoughts by watching the breathing. We can't expect alert and balanced
intelligence from an agitated, excited mind, and so estabshing at least
some calmness is essential.

The calm mind is suitable for investigation. What can a happy person
learn from the teaching of the Buddha? What can an unhappy person learn?
Or, when nothing special is going on with someone, just feeling okay?

We are observing our experience, the signs of impermanence, the
beginning and ending of feelings and thoughts, how they appear, change
and disappear. The understanding of the Four Noble Truths is connected
to reality through this investigation for ourselves, the teachings we
heard acquire meaning this way.

Our experiences manifest through the senses, in the form according to
the sense. Forms and colours are perceived by the eye, sounds by the
ear, smells by the nose, tastes by the tongue, touch, hot and cold by
the body, and the mind shows toughts, memories, mental processes.

These appear in three qualities -- they may be pleasant, and we are are
attraced to them, they may be unpleasant and we would rather distance
ourselves from them, or they may be neutral, and their presence doesn't
bother us, but if we pay attention to them they may be pleasant, like
the breathing.

The appearance and cessation is not in our direct control, the necessary
conditions are simply that the sense makes contact with the sense-object
appropriate to it, and our attention should be directed there. The
sensation appears on its own. When the contact is broken, or our
attention turns somewhere else, the sensation disappears.

The happy person -- who is experiencing pleasant sensations, can learn
from this to not believe in the attractive impression, and not to cling
to the pleasant sensation, because this dependent condition is
unreliable. It is not their own, not possible to keep, has no essence
and is empty.

The unhappy person -- who experiences unpleasant, painful sensations,
can learn that this condition is not going to last, and not to be taken
with anger or hatred because of this, it is enought to wait with
patience.

The person who feels like living in a neutral, grey world, can learn not
to give himself over to carelessness and foggy confusion, because this
neutral condition is not going to be permanent either, and if they
follow a mistaken view through lack of alertness, it will result in
disappointment, pain and suffering.

The impermanence and emptiness fundamentally changes our view, it
reorganizes our values.

When many things are going on in our head and the thoughts are not
slowing down, we can occupy the thinking with a thought which we
determine in advance, instead of allowing it to run in every direction.
The BUD-DHO mantra is useful in that case, it collects the scattered
attention with a simple method.

We simplify the meditation down to the essence, numerous complicated
steps only increase the sense of unfamiliarity and doubt.

One breath, one BUD-DHO. On the in-breath, we internally recite the
first half of the mantra, BUD-, the breath pauses in the middle for a
moment, and on the out-breath we recite the other half, -DHO. BUD-DHO.

The essence is peace, and the understanding, which stops you. The peace
originates from the senses drawing in and turning inwards. The seeking
stops, because what is here is enough, and there is no need to go
anywhere. The quiet joy arises from the mind which understands that
there is no happiness in the world to be pursued. The values reorganize
themselves, we don't look for the strength and happiness, because this
dependent condition is always uncertain and exhausting.

Where is the peace now? Where is the understanding now? There is nothing
to solve. BUD-DHO, a few breaths, and the stories of the world are not
interesting for us. It is enough, when the question stops to mind. This
pause is the listening silence, and the answer is not necessary.

In an everyday situation, simplify it down, until it is clear to
recognize. Whether with a mantra, or wordlessly. You are completely
exhausted, you have no energy, your mind is spinning with the coming and
going of the day, but the breath is still available, the silence can be
still felt.

In what kind of situation can we expect to learn something we didn't
understand before? Looking back, I remember when everything was going
well and kept under control, at best I could repeat the old. When I was
feeling terrible, sorry for myself and complaining, I really didn't
learn anything from that, and when I followed things according to habit
just like yesterday, that wasn't particularly useful either.

So we are not seeking the feeling themselves, we are not trying to
create special feelings by meditating, we are not looking for the
situation when everything will be always pleasant. The pleasant,
unpleasant, neutral feelings in themselves will not give us right
understanding, because we only follow their influence and react to them
mechanically -- the awareness has to notice their impermanence and
uncertainty, then we can see with understanding what is wholesome, what
is unwholsome in the present situation.

Is the practice easy or difficult? A useful image to think about is how
a boat moves on the river. When the boat is full of seep-water, or
packed with crates filled with goods, the boat is moving slowly. It is
burdened, heavy and slow, all the seep-water and crates are weighing it
down.

We would like our boat to go fast, don't we? But at the same time we are
holding onto everything we've put on it. We have to lighten the boat,
let go of the self, which is the heaviest burden. We create the load of
'me' and 'mine', we create the impression of 'I have been like this, I
am like this, I should be like this', 'this was mine, this is mine, this
I have to get'. This is what is holding to boat down, this is the
weight.

The boat is empty, when it is empty of me and mine. It moves swiftly,
because it is light as a feather, not heavy with self, the stories and
drama of 'me' and 'mine'.

What happens, if we are sitting in a boat, and somebody runs into us
with their boat? We angrily push them away with the oars, perhaps even
shout at them. What happens if an empty boat runs into us? Where did the
eariler anger and emotion come from?

We have a tendency to manufacture stories about me and mind, whether
based on real or imagined events. If we take them seriously, and give
them reality, the stories start to control us, and we create problems
which didn't exist before.

In the practice of meditation, we restore right view by returning to the
simplicity of the senses. and the stories, if there are, we can see them
not from a fixed perspective. By investigating the senses, we take a
more fundamental level as the base. Pleasant feeling is like this, as we
are experiencing it, unpleasant feeling is like this, neutral feeling is
like this, it has a beginning and an end, changing and empty.

In the practice, the value is not going to be in accumulating and hurry,
but in leaving space for letting go and patience. There are times for
action, but many difficulties are solved by simple patience. The sense
of being hurt and importance comes from ourselves, at such times
restraint is a safe perspective toward ourselves and others, and silence
is enough.
