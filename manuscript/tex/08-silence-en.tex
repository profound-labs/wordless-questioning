\chapter{Silence}

\section{Signal}

\keywords{relation with our surroundings}

Two of us are walking down a path leading to the entrance of the
monastery. We are talking about one thing or another, but as we enter,
we notice the silence and our conversation stops. The Dhamma hall is
beyond the next door, and we don't want to disturb anyone who might be
meditating inside. We close the door quietly behind us. We are going
somewhere else in the building, but the significance of the Dhamma hall
is above that mundane task.

The silence of listening creates an implied relation with our
surroundings. In the previous case with the person who could be sitting
in the Dhamma hall, but even if we could see that there is nobody there,
we would still lower our voices or stay silent. When we enter, the
silence serves as a signal to pay attention. We give space for the
values beyond ourselves represented by the Dhamma hall dedicated to the
truths of the heart and mind.

In this context, silence is a signal which directs us to remember that
which is beyond the worldly values. When we enter a church, monastery or
other sacred spaces, we look beyond noisy worldly affairs and beyond our
usual preoccupation with ourselves.

We have enough experience of noisy chatter to know that profound
comprehension is not found there. So we fall silent to give our
attention to listening, to be a part of the understanding we cannot
express in words. We move in silence, we listen in silence, carefully
keeping ourselves out of the way, so that we may hear the message of the
place and let the activity speak for itself. This silence is a presence,
not isolating, but including the space and the other beings who live
there. In the words of David Whyte,\footnote{\href{https://www.goodreads.com/quotes/10119971-the-winter-of-listening-no-one-but-me-by-the}{The
  Winter of Listening by David Whyte}}

\begin{quote}
You can belong\\
to everything\\
simply by listening.
\end{quote}

\section{Valuable}

\keywords{paying attention, serenity in silence}

Silence also expresses how much we value what we are doing. Being silent
and maintaining attention is an expression of alertness and respect for
that activity. This is both an inward and outward signal: Others see
that whatever we are doing, it requires silence. We also see ourselves
being silent, voluntarily restraining our impact on ourselves and on our
surroundings, which communicates that where we are and what we are doing
has more significance than chattering about ourselves.

Calmness, understanding and silence are closely connected. We stop
speaking and pay careful attention to investigate and understand a
phenomena. After this verbal silence, the mind continues, `Why? Why?',
but when the penny drops at the `Aha!' moment, the mind also stops
chattering and we are silent, feeling glad for the understanding. In
that content and serene mood, we stay silent, for the moment nothing
further is required.

\begin{quote}
On hearing true teachings\\
the hearts of those who are receptive\\
become serene,\\
like a lake, deep, clear and still.

\bigskip

\quoteRef{%

Dhp 82.\footnote{\href{https://forestsangha.org/teachings/books/a-dhammapada-for-contemplation?language=English}{A
  Dhammapada for Contemplation by Ajahn Munindo (forestsangha.org)}}

}
\end{quote}

\keywords{lying down meditation method}

Stillness is most characteristic in the lying down meditation posture.
In this posture the muscles of the body are completely relaxed, which
creates a sense of ease and comfort, although one has to take care and
avoid falling asleep. When you are physically tired, this posture is not
recommended for meditation, but rather the sitting, walking or standing
postures, which raise the body's energy level by the physical effort.

Establish a clear intention to be wakeful. Before lying down, it sets
the right attitude and separates us from the daily activities, if we
first bow to a Buddha shrine, and softly recite a short chant.

\clearpage
\thispagestyle{empty}\mbox{}
\photoFullBleedPlaceholder{TODO: Illustration of lying down meditation.}
\clearpage

A yoga mat or soft carpet is useful to avoid getting sore when lying on
the floor. If you feel the breathing becoming obstructed as the head
inclines backward, use a small, firm pillow or a folded towel to prop up
your head. Lying down in a bed could be too soft, and it reminds the
body about sleeping. Let the hands rest at the side of the body. If you
place them on the belly or the chest, the rising and falling movement
can be distracting. Pull up the knees, so that the feet can be flat on
the mat. This avoids tension in the joints and helps to maintain
alertness.

Maintain this posture while relaxing the muscles in the body and
cultivate physical stillness. Direct attention inward and use the
sensation of breathing as your meditation. Experiment with the breath,
use it to brighten the mind and maintain clear comprehension. If the
mind drifts into dull greyness, drowsiness will follow. Setting a timer
could be appropriate, either to signal the end of the session, or as a
periodic reminder to remain alert.

\keywords{noise exposure, available cognitive capacity}

Speaking of the advantages of silence is not to say that sound cannot be
pleasant. Music clearly has therapeutic effects, and helps to relax an
agitated mind. It can be \emph{very good music} (in our opinion), but
how many times in a row can you listen to it? One single thing over and
over, it turns from pleasant to painful in a short time. Have you
listened to music for hours, thinking, `I like this music', but still
feeling relieved when you turned it off? `Good song, but I've been
missing this silence.'

Sounds are input signals which stimulate the nervous system, it can feel
good for a while, but it is still a continuous stimulation. Noisy
environments degrade attention and intelligence, you can probably
remember how hard it is to think clearly when there is a construction
site at your neighbour's property. Apart from personal experience,
medical studies have also measured how `mental workload and visual /
auditory attention is significantly reduced'\footnote{\href{https://www.ncbi.nlm.nih.gov/pmc/articles/PMC6901841/}{The
  Effect of Noise Exposure on Cognitive Performance and Brain Activity
  Patterns (ncbi.nlm.nih.gov)}} when being exposed to noise.

Mobile phones don't even have to make a noise to cause `brain drain':
another study found that `the mere presence of one's own smartphone
reduces available cognitive capacity'.\footnote{\href{https://www.journals.uchicago.edu/doi/10.1086/691462}{Brain
  Drain: The Mere Presence of One's Own Smartphone Reduces Available
  Cognitive Capacity (journals.uchicago.edu)}} It is not surprising that
traditional insight meditation retreats try to setup a silent
environment and ask participant to not bring their mobile phones to the
meditation hall, or leave them in a locked place for the entire retreat.
Give that nervous system a break and let it settle, don't let it be like
the crow in Santoka's haiku,\footnote{\href{https://www.goodreads.com/book/show/931086.Grass_and_Tree_Cairn}{Grass
  and Tree Cairn, Taneda Santoka}}

\begin{quote}
Cawing a crow,\\
flapping a crow,\\
with no place to settle down.
\end{quote}

\clearpage

\section{Chanting}

\keywords{setting a clear intention, wholesome thoughts}

At the monastery, we practise chanting before or after the daily group
meditations. First, when we arrive individually to the Dhamma hall, we
bow three times in silence toward the Buddha shrine. The senior monk
rings a small bell to signal the start of the chanting. We wait in
silence while he lights the candles and incense on the shrine, and we
bow again. During the bowing, there is always silence.

We begin the chanting together, synchronizing our voices: too soft can't
be heard, too loud is harsh and being out of tune, separates from the
harmony. (A good guideline is that if you don't hear your own voice, you
are chanting too softly, and if you can only hear yourself, you are too
loud.) The text of the chants are recollections of the Buddha and the
teachings, this is a practice of directing the mind toward thinking
wholesome thoughts. This ordered, symbolic ceremony uses a rhythm of
speech and action as a skilful tool to clear the mind before the silence
of meditation.

The exact routine varies between monasteries. The morning meditation at
Sumedhārāma in Portugal starts at 5am, when we start with an hour of
sitting meditation, during which there is no speaking or chanting. When
you enter, there is silence, a shared space for inner reflection for an
hour, until the senior monk rings the bell at the end of the meditation,
which is followed by 15-20 minutes of chanting.

\clearpage

\keywords{boredom, learning about the mind, inner peace, sense-restraint}

`Doesn't it get boring?' From time to time, a school program brings a
whole class of children to meditate in silence (perhaps hoping that they
become more quiet afterward), and they probably do feel bored out of
their skull. They had no interest to be there from the outset, but
children are clever and they tolerate the strange ideas of the adults.

Boredom changes as soon as you look at it. When you come to meditate,
you are interested in learning about yourself and your mind, and looking
at it closer, `boring' becomes rather interesting. `Not much is
happening, just the breathing. Is that a problem for me? \emph{Am I}
creating that problem? Can I stop creating problems for myself? The
breathing has a subtle richness, a pleasant feeling actually.'

When practising mindfulness of breathing, there is a gladness born of
sense restraint. The mind relaxes, and the thinking can be allowed to
stop. We are silently observing experience, there is no need to comment.

Boredom is a combination of factors: the desire for excitement, active
dismissal of the present, and the attitude that we already know, there
is nothing new here. Is it not intrinsic quality of the situation, but
rather a habit of the untrained and restless mind. The Buddha compared
it to how an elephant feels when the trainer first restrains him by
tying him to a strong post. The elephant is unfit to train while he
keeps longing to wander in the wilderness as he wishes, but a good
trainer gradually restrains his restlessness until he learns to remain
content.\footnote{\href{https://suttacentral.net/mn125}{MN 125}, The
  Level of the Tamed} In the sutta, Prince Jayasena doesn't even believe
that inner peace is possible through sense-restraint, since he lives in
a palace surrounded by distractions and never experienced with such
peace.

In the meditation hall the door is open, you can stand up and walk out
any time. But you are there because your mind have wandered as you
wished before, but felt that it was ultimately unsatisfactory, and the
untrained mind kept making painful mistakes and trouble. If you walk a
thousand steps in a thousand directions, you just get tired, and angry
why you didn't get anywhere. We pass a threshold when we recognize the
need to be the trainer of our restless mind. We learn what the right
direction is and keep stepping that way.

\begin{quote}
For the mind that is difficult to subdue,\\
flighty, flitting wherever it will,\\
restraint is good,\\
a restrained mind brings happiness.

\bigskip

\quoteRef{%

Dhp 35.\footnote{\href{https://www.ancient-buddhist-texts.net/English-Texts/Dhamma-Verses/03-Mind.htm}{Dhammapada,
  translated by Ānandajoti Bhikkhu}}

}
\end{quote}

\clearpage

\section{Shrine}

\keywords{creating sacred spaces, symbolism of a Buddha shrine}

I didn't always have a Buddha shrine in the room or hut where I stayed
at the monastery. I thought they were part of conforming to
institutional expectations. So I mostly ignored and subtly resented
pictures and statues, because I felt other people expected me to
venerate them, and spitefully I wasn't going to do what (I thought) they
expect. My reaction was like that of school kids: I was clever enough to
tolerate the symbols and arrogant enough to think I already know what
they mean. Believing oneself clever makes one feel superficially
dismissive and bored with everything. It is a self-stupefying
combination, thinking that \emph{I know} closes our mind, so we can't
learn that we don't know. The British psychologist Iain McGilchrist
compares it to being stuck in a labyrinth of mirrors:\footnote{\href{https://www.goodreads.com/book/show/6968772-the-master-and-his-emissary}{The
  Master and His Emissary: The Divided Brain and the Making of the
  Western World by Iain McGilchrist}} all you can see is what you tell
yourself, and never find a way out.

A small crack must have appeared on those mirrors, because I noticed
that in fact, nobody was making such judgment and expectation of me. I
was creating both sides of the story and I felt consumed over something
I only imagined.

Making a Buddha shrine opens up a small space in the place we live, a
reminder to stop the hurry and make space for awakening. The Buddha
shrine in the meditation hall gives us the same message through silence.
A shrine is a gift for us from ourselves. It is not for answering other
people's expectations, or even for the Buddha. The historical Buddha
passed away 2600 years ago and is beyond expecting or needing anything
from us. Other people have enough to worry about and don't think about
us as much as we assume.

I remember thinking, `Why don't I have space for the Buddha in the place
where I live?' Then I started to cut some wood planks and made a small
shelf for the shrine. Buddha shrines are often quite plain: one or more
Buddha figures, candles, incense and flowers. The Buddha represents
awakened consciousness in the human form. The candles are for wisdom
which makes things visible, as the light in darkness. The incense refers
to a saying of the Buddha about virtue, `the fragrance of virtue
pervades all directions' (Dhp 54). The flowers are a symbol of virtue,
happiness and impermanence. They are like our practice: they bring
happiness if we take good care of them and renew them frequently, while
their fading is a reminder of transience.

I offer these words of reflection with the intention that they may
encourage your practice. The teacher is the Buddha, the source of
illuminating explanations leads back to him. I am grateful that his
teachings have been carried on through the centuries by each generation
to this day. May they help turning the noisy confusion in our mind into
understanding silence.
