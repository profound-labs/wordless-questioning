\chapter{Csónak}

A meditáció elején a légzés figyelésével ellazítjuk a testet és
lecsillapítjuk a gondolatokat. Egy zaklatott, izgatott elmétől nem
várhatunk éber és kiegyensúlyozott intelligenciát, ezért legalább némi
nyugalom megalapozása lényeges.

Vizsgálódásra a nyugodt elme alkalmas. Mit tanulhat a boldog ember a
Buddha tanításából? Mit tanulhat a boldogtalan ember? Vagy, akivel semmi
különös nincs, éppen megvan ahogy van?

A tapasztalatunkat figyeljük, az állandótlanság jeleit, az érzések,
gondolatok kezdetét és végét, ahogy megjelennek, változnak és
megszűnnek. A magunk számára vizsgálódunk, ez ad a tanítások szavainak
jelentést és közvetlen hasznot.

Tapasztalataink az érzékeken keresztül nyilvánulnak meg, az érzéknek
megfelelő formában. A szem formákat és színeket fog fel, a fül hangokat,
az orr szagokat, a nyelv ízeket, a test a tapintás, a hideg és meleg
érzett benyomásait érzékeli, az elme pedig a gondolatokat, emlékeket,
mentális folyamatokat jeleníti meg.

Ezek három minőségben jelennek meg -- lehetnek kellemesek, és vonzódunk
feléjük, lehetnek kellemetlenek vagy fájdalmasak, és inkább távolodnánk
tőlük, vagy lehetnek semlegesek, és a jelenlétük nem zavar minket, de a
semleges kellemes érzéssé válik, ha figyelünk rá, mint a légzés.

A megjelenésük és elmúlásuk nincs közvetlen irányításunk alatt, a
szükséges feltételek a kapcsolat az érzék és érzék-tárgy között,
illetve, hogy a figyelmünk oda irányuljon. Az érzet magától megjelenik.
Amikor a érzék-kapcsolat megszakad, vagy a figyelmünk másfelé fordul, az
érzet megszűnik.

A boldog ember -- aki kellemes érzéseket tapasztal, ebből azt
tanulhatja, hogy ne higgyen a vonzó benyomásban, és ne ragaszkodjon a
kellemes érzéshez, mert ez a függő állapot megbízhatatlan. Az érzet nem
tartozik őhozzá, nem megtartható, nincs mélyebb lényege és `én' nélküli,
üres.

A boldogtalan ember -- aki kellemetlen, fájdalmas érzéseket tapasztal,
azt tanulhatja, hogy ez az állapot nem lesz maradandó, és fölösleges
ezen felhúznia magát haraggal vagy gyűlölettel, mikor elegendő
türelemmel várnia.

Aki úgy érzi, semleges és szürke világban él, azt tanulhatja, hogy ne
engedjen emiatt a figyelmetlenségnek és ködös zavarodottságnak. Ez a
semleges állapot sem lesz állandó, és ha az éberség hiányában téves
nézetet követ, az eredmény fájdalmas és veszélyes lehet, mintha falnak
rohannánk vagy gödörbe esnénk a ködben.

Az állandótlanság és üresség alapvetően megváltoztatja a nézőpontunkat,
átrendezi az értékeinket.

Ha veszély van, vagy a helyzetünk bizonytalan, természetes, hogy arról
gondolkodunk mit kellene tennünk, a félelem és szorongás érzése meg fog
jelenni, mert jó ok van rá. A félelem, mint érzelem, a lehetséges
veszély információját hordozza, a szorongás, mint érzelem, egy
bizonytalan eredményt foglal magában. A félelem óvatossá tesz minket,
ami hasznos -- nem szeretnék egy autóban ülni olyan sofőrrel, aki nem
fél az ütközéstől.

Mit várhatunk el a meditációtól? Azt gondolhatjuk, \emph{ha jól tudnánk
meditálni}, meg tudnánk állítani a félelmet és szorongást, alkalmazzuk a
megfelelő technikát vagy felidézzük a megfelelő szavakat és el kellene
tűnnie. Vedd észre ebben a motivációban az irányításra törekvő vágyat.
Azt kívánjuk, hogy a helyzetünk más legyen, mint amilyen, láthatjuk a
vágyat ami manipulálni és megszüntetni akarja.

A szorongást követő belső párbeszéd a körülöttünk történő események
irányítására törekvés egy formája, újra-értelmezni amit látunk olyan
módon, ami illeszkedik a helyzetről alkotott korábbi nézetünkhöz. Amikor
az érzés már megjelent, nem tudjuk megváltoztatni vagy kijavítani, de
továbbra is részünk van a folyamatban -- a hozzáállásunk befolyásolja
milyen irányba fejlődik az érzés, rosszabbíthatunk rajta, vagy teret
adhatunk neki, hogy kifussa magát és véget érjen.

Amikor a reptéren a csomagomra várok, érzem a szorongást -- vajon
elvesztették a csomagom? Megettem mindent amit tennem kellett, és most
semmit többet nem tehetek. Érzem a szorongást, mert a csomagom helyzete
valóban bizonytalan. A gyakorlás részeként felidézem, hogy tudok helyet
adni ennek az érzésnek és vele maradni, nem kell siettetni, addig
maradhat ameddig maradnia kell.

Nem tudjuk megállítani, de megállhatjuk, hogy rontsunk rajta. Ha veszély
van, megtesszük ami szükséges; ha nincs veszély \emph{pillanatnyilag},
de szorongást érzünk, felismerhetjük, hogy \emph{nem a szorongás a
veszély}, és éberen jelen maradunk, az érzéstől való félelem nélkül.

Hogyan észlelhető számunkra az érzés, a testen keresztül szemlélve? Hol
érezzük? Mikor kezdődött? Változásban van? Valóban olyan rossz, mint
testen belüli érzés? Ez nem ad nekünk irányítást, de fejleszti a
megértést, hogy nem az érzés a veszély, és nem kell folytatnunk a belső
küzdelmet az irányításért.

Ha a gondolatok nem csillapodnak, lefoglalhatjuk a gondolkodást egy
előre eldöntött gondolattal, ahelyett, hogy engednénk minden irányba
rohanni. A BUD-DHÓ mantra hasznos ilyenkor, összefogja a szétszórt
figyelmet egy egyszerű módszerrel.

Ha úgy érzed a meditáció túl bajjal jár, egyszerűsítsd le a lényegre,
sok bonyolult lépéstől csak növekszik az ismeretlenség és kétség érzete.

Egy lélegzet, egy BUD-DHÓ. Belégzés közben magunkban szavaljuk a mantra
első felét, BUD-, középen a lélegzet megáll egy pillanatra, és kilégzés
közben a másik felét szavaljuk, -DHÓ. BUD-DHÓ.

A lényeg a béke, és a megértés, ami megállít. A béke abból ered, hogy az
érzékek visszahúzódnak és befelé néznek. A keresés megáll, mert ami van
az elég, sehova nem kell mennünk. Csendes öröm jelenik meg az elmében,
amikor megértjük, hogy a világi célok üresek és nem szükségesek, a
folyamatos éberség felismeri a jelenben lévő boldogságot. Az értékek
átrendeződnek, nem kívül keressük az erőt és biztonságot, mert a függő
feltételek bizonytalanok, elégtelenek, és hajszolásuk vég nélkül
fárasztó.

Hol van a béke most? Hol van a megértés most? A tapasztalat nem egy
megoldani való probléma, a tudatos figyelem vele marad és felfogja.

Fordítsd a figyelmet a kérdés előtti pillanatra: ki kérdez kit? Ez a
narrátori elme trükkje, elképzel valakit, akihez beszél, valakit akit
kritizál vagy panaszkodik neki, de a mikrofonba beszélő hang és a
hallgató ugyanaz, és a kérdés és válasz között egyik sincs.

BUD-DHÓ, belégzés, kilégzés, és a világ történetei már nem érdekesek
számunkra. Amikor a kérdező figyelem megállítja a szavakat az elmében,
ennyi elég. Hallgató figyelem tölti be a szünetet, és a válasz a jelen
tapasztalat.

A lélegzeten és BUD-DHÓ mantrára épülő meditációt könnyű informális
helyzetekhez is igazítani. Hétköznapi helyzetben, egyszerűsítsd le, amíg
megfelelő hozzáállás tisztán kivehető, akár egy mantrával, akár
szótlanul is. Még egy zsúfolt nap után is, mikor kavarog a fejed a napi
jövés-menéstől, a légzés akkor is elérhető, a csend ott is érezhető.

A tanulás mint ötlet vonzó, de milyen helyzetben várhatjuk azt, hogy
tanulunk valamit, amit korábban nem értettünk? Visszanézve arra
emlékszem, hogy amikor minden jól ment és irányítás alatt volt,
legfeljebb a régit tudtam ismételni. Amikor szörnyen éreztem magam és
keseregve panaszkodtam, abból végleg nem tanultam semmit, és amikor
mindent szokás szerint követtem a triviális, kényelmes szokásokat nap
mint nap, az sem volt kifejezetten hasznos.

A csendes és nyugodt időszakok áldásnak számítanak, mindig is értékeltem
egy stabil rutint, ami hosszú időszakokat enged a koncentrált munkára
vagy intenzív gyakorlásra. Más részről, akadályok és konfliktusok is
garantáltan jönni fognak, szándékosan szembe nézni és jó képességgel
megbirkózni ezekkel arany esélyt ad arra, hogy az elme képzését a
korábbi korlátokon túl fejlesszük, és a zavaros káoszból tudatosan
gyakorlati hasznot teremtsünk.

Nem magukat az érzéseket keressük, nem különleges érzéseket próbálunk
létrehozni a meditációval, nem azt a helyzetet keressük, ahol mindig
minden kellemes. A kellemes, kellemetlen, semleges érzések önmagukban
nem adnak nekünk helyes megértést, ha követjük a befolyásukat és
gépiesen reagálunk rájuk -- az állandótlanságukat, bizonytalanságukat
kell az éberségnek észrevennie, akkor megértéssel látjuk mi a jótékony,
mi a kártékony a jelen helyzetben.

Könnyű a meditációt gyakorolni, vagy nehéz? Egy hasznos kép amire
gondolhatunk, ahogy egy csónak halad a folyón. Amikor a csónak áruval
töltött ládákkal van megrakva, a sok teher alatt nehezen és lassan
halad, épp, hogy a víz felett tudja tartani magát.

Azt szeretnénk, hogy a csónakunk gyorsan haladjon, nem igaz? De
ugyanakkor ragaszkodunk mindenhez amivel megraktuk. Könnyítenünk kell a
csónakon, elengedni az `én' nehéz súlyát. Mi hozzuk létre az `én' és
`enyém' terhét, mi hozzuk létre a benyomást, hogy `ilyen voltam, ilyen
vagyok, ilyen kell legyek', `az az enyém volt, ez az enyém, ez meg
akarom tartani, azt meg kell szereznem'. Ez a súly húzza le a csónakot.

Az érzés, hogy elég, ami van, megnyitja a nagylelkűséget. Az
elégedettség a gyakorlás folyamatos része, nem egy rögzült, feltételhez
kötött állapot. Tettek és tanulás könnyű áramlatban folynak az
elégedettségből -- mikor azt gondolom, `kész leszek elkezdeni, mikor már
megvan a \ldots{}', az elégedetlenség köti le a gondolataimat és folyton
megszakítja a koncentrációmat a jelen helyzetre. Viszont amikor azt
gondolom, `nem vagyok jó ebben, de ennyi is elég, hogy elkezdjem',
elfogadni a jelen határaimat energiát ad a cselekvésre, és gyakran
meglepődök, hogy többet tudtam tenni mint képzeltem.

A gondolkodás rossz hírnevet kap a meditációs könyvekben, de a tiszta
gondolatok megalapozzák a feltételeket a helyes hozzáállás fejlődéséhez.
Az elharapózó, kényszeres gondolkodás valóban fájdalmas tapasztalat, de
arra törekedni, hogy megállítsunk minden gondolatot, is mellé lő a
hasznos célnak. Vedd észre, hogy a jótékony gondolatokat megelégedettség
és béke követi. Erkölcsös tetteket tudatosan felidézve kialakul a
stabilitás és ön-tisztelet érzete, és bízunk magunkban, hogy elengedjük
ami fölösleges, mert érezzük, hogy ami van az már elég.

Ha fejben akarjuk megoldani, a gyakorlás gyorsan bonyolulttá válik. A
meditációban, a testen keresztüli éberség egy megbízható irány, az
érzéseket és elme állapotokat figyeljük, ahogy jönnek és mennek, ez egy
más nézőpontba helyez minket. A bonyolult kérdéseket magunk mögött
tudjuk hagyni, mert már nincs szükségünk a válaszokra.

Mi teszi lehetővé, hogy tovább tanuljunk és fejlődjünk? Az utazás akkor
a legélvezetesebb, amikor a horizont a korábbi korlátainkon túl egyre
tovább tágul. A horizontot nem azzal tágítjuk, hogy messzire utazunk,
hanem, hogy új szemmel látunk; a vágy, hogy megtartsuk azt, amiről azt
gondoljuk mi vagyunk, határozza meg a jelen korlátunkat.

A csónak könnyű, amikor üres az éntől és enyémtől, nagy távolságokat
tesz meg dráma és zaj nélkül. Mi történik, ha a csónakban ülünk, és
valaki a csónakjával nekünk ütközik? Rákiáltunk, ellökjük az evezővel,
és egész nap erről panaszkodunk. Ez mind lehet, hogy jogos, de tönkre
tettük a napunkat a saját rossz társaságunkkal, nehéz ebben a
bölcsességet látni. Mi történik ha egy üres csónak ütközik nekünk?
Honnan eredt a korábbi harag és indulat?

Hajlamosak vagyunk én és enyémről szóló történeteket gyártani, akár
valós, akár képzeletbeli események alapján. Ha komolyan vesszük ezeket,
és valóságot adunk nekik, a történetek kezdenek irányítani minket, és
korábban nem létező problémákat hozunk létre.

Előfordul, hogy ülünk a meditációs párnán, és vitákat kezdünk eljátszani
a képzelet bábjaival. Komoly az ügy, nekünk kell nyerni! Módszeresen
végig gondolni egy problémát hasznos eszköz, de önmagunk felé irányuló
szimpátiára és kedvességre is szükség van az építő jellegű belső
bárbeszédhez. Másként, amikor az én magával beszél, rossz társaságban
találja magát.

Meglepő, mennyire fel tudjuk húzni magunkat egy olyan helyzettel
kapcsolatban, ami még meg sem történt. Segít, ha tartunk egy csipet
humort az oldal zsebünkben, vészes komolyság esetére. A görög filozófus,
Epiktétosz mondását felidézve, `Aki tud önmagán nevetni, sosem fogy ki a
nevetni valóból.'

A meditáció gyakorlásában visszaálltjuk a helyes szemléletet azzal, hogy
visszatérünk az érzékek egyszerűségéhez. A történeteket, ha vannak, a
változó körülmények nézőpontjából szemléljük. Az érzetek vizsgálatával
egy alapvetőbb valóságot veszünk alapul. A kellemes érzés ilyen, ahogy
most tapasztaljuk, a kellemetlen érzés ilyen, a semleges érzés ilyen,
kezdete van és vége, változik és üres.

A gyakorlásban nem az lesz értékes, hogy sietve eredményeket halmozunk
fel, hanem, hogy teret hagyunk az elengedésnek és türelemnek. Van amikor
cselekedni kell, de meglepően sokféle nehézséget megold az egyszerű
türelem. A sértettség vagy sürgető fontosság érzése magunkból ered, a
visszafogottság a magunk és mások irányába biztonságos nézőpont
ilyenkor, ami hagyja a csónakot csendben tovább úszni.
