\chapter{Ciklusok}

A meditáció a jelenbeli érzéseken, tapasztalatokon keresztüli
megismerést tanítja. Az utasítások lépésről-lépésre írják le a
fejlődést, de a jelenben csak egy pillanat elérhető számunkra. Előre
vagy hátra lépünk? Bármely irányban, a tapasztalat egyszerre egy lépés,
a lépés ahol minden változik. Szavakat használunk, hogy leírjuk a
tapasztalatot, de a tudatosság erre a tapasztalatra szótlan, miközben az
éber figyelem aktívan befelé néz, mintha önmagát kérdezné. Még a nyelv
szimbólumai is korlátozóak, mert rögzített reprezentációkból állnak, míg
a tapasztalat mozgásban van.

A meditáció célja nem az utasítás lépéseinek tökéletesítése. A cél a
jelen tapasztalat tiszta ismerete, a helyes nézőpont visszaállítása.
Kialakulhat bennünk az a benyomás, hogy mindig ugyanazt a lépés sort
kell teljesítenünk, és amikor az elme nem aszerint a sorrend szerint
fejlődik, csalódottak vagyunk.

És ami még rosszabb, úgy látszik mások békésen meditálnak, ők biztos jól
értik! A gyakorlás felszínre hozza az ön-kétséget. Vizsgálhatjuk ezt a
másik oldalról: Valaki talán dicsér minket, `Olyan békésnek tűntél, te
biztos tudsz valamit!' De mi tudjuk milyen szétszórt gondolatokkal volt
tele a fejünk -- ennyit az ilyen vélemények igazság értékéről. A
narrátor elme gépiesen folyton megjegyzéseket tesz, de nincs mögöttük
mélység vagy vizsgálódás, még amikor azt gondoljuk is, `ebben biztos
vagyok'.

Fordítsd meg ezt a hozzáállást és kezd a tapasztalattal, egy kérdező,
kíváncsi figyelemmel ami szótlanul érdeklődik a jelenről. Ha a
tapasztalat az alap, ahogy a tapasztalati most van, az milyen megértést
ad nekünk? Először magunkat vesszük szemügyre, hogyan érezzük magunkat,
milyen állapotban vagyunk, és erre válaszolunk intelligensen, a
meditációnk megfelelő irányba való fejlesztésével. Amikor falat festünk,
először a falat vesszük szemügyre, kiválasztjuk az annak megfelelő
festéket, és \emph{azután} követjük az utasítást a dobozon. A rossz
fajta festék le fog peregni, nem igaz? Van amikor le kell higgadnunk,
máskor energiát és erőfeszítést kell bevetnünk, vagy várni, hogy a belső
vihar tovább álljon.

A lépések az imitációs tanulás részei, egy példát követve figyeljük
önmagunkat és meglátjuk hogyan működik ez. Amikor szenvedést érzünk, és
vagy fel tudjuk oldani, vagy türelmesen kivárjuk, amíg véget ér. Tiszta
fejjel visszanézünk, tudjuk mi történt minek a hatására, és a
gyakorlásba vetett bizalmunk erősödni fog. Ebből tanultunk valamit, és
nem ragaszkodunk az első, bemutató példa részleteihez.

Ez egyszerű lenne, ha a meditációnk egyenes vonalban fejlődne, a
gyakorlással töltött percek és órák számával egyenes arányban. Úgy
képzeljük, hogy le fogunk ülni, az elején kissé szétszórtan, de egy
órával később, \emph{ha jól tudunk meditálni}, nyugalmat fogunk érezni,
az elménk tiszta és összeszedett lesz. Legalábbis erre számítunk.

Később vissza emlékezünk mi történt a meditáció alatt, azt látjuk, hogy
nem ez szokott történni. A tapasztalat nem egyenes vonalban fejlődik a
sekélytől a mélyig, vagy a szétszórttól az összeszedettig. Azt
gondolhatjuk, ez a mi hibánk, mert nem vagyunk jók a meditációban, vagy
nem helyesen követjük a lépéseket.

Amit megpróbálunk terv szerint, lépésről lépésre haladni, minden
különbözik az elvárásainktól. Arra gondolhatunk, talán, nem próbáljuk
elég erősen? Egyre jobban neki feszülünk, és egyre fájdalmasabb lesz.
Ilyen az az érzés, amikor egy véleményt rá akarunk erőltetni a
tapasztalatra.

Ha felidézzük, hogy a tapasztalatunk hogyan változik idő közben, más
mintát látunk. Egy tapasztalat megjelenik, változáson megy keresztül,
elmúlik, és egy újabb tapasztalat jelenik meg. Az elme ilyen ciklusokban
fejlődik, és ezek a ciklusok nem vesznek tudomást a céljainkról, hogy a
meditációnkat úgy akarjuk fejleszteni mint egy rang létrát.

Vehetjük a tapasztalatot alapigazságnak, és onnan indulunk. Milyen fajta
tapasztalat ez itt? Észrevehetjük, hogyan mozog a figyelem, mint a
tudatosság egy folyamata, hogyan megy keresztül különféle ciklusokon.

Eleinte az elme elégedett az üléssel, ellazult figyelemmel pihen, mint
amikor egy séta után leülünk egy padra, ülni és lélegezni a béke
teljessége. Gondolatok jelennek meg és követjük őket. Megállunk, újra
nyugodtak és csendesek vagyunk, a gondolkodás lehet, hogy meg is áll
anélkül, hogy észrevennénk, hogy nem gondolkodunk. A figyelem elkezd
mozogni, és megint észrevesszük magunkat, hogy gondolkodunk. Emlékek,
vágyak, nyugtalanság jelenik meg és észrevesszük, hogy ezen dolgoznunk
kell. Ezután az elme újra nyugodt, és visszatér a csendesség érzéséghez.

Némi ismeretre szükség van, de egy kevés is elég. A Buddha tanítására
emlékezni olyan kincs, ami nem fogy ki. Ez a tudás viszont nem válik a
miénkké, nem tehetjük egy dobozba, hogy eltárazzuk a következő
alkalomra. Bármit is tanultunk, minden alkalommal újból az elejénél
kezdjük, és onnan, bízunk a jelen megismerésében.

A gondolkodó elme vonzódik a tényekhez és megállapításokhoz, egyfajta
biztonságot érzünk abban, ha tényeket tudunk felmondani. Szeretnénk
megállapítani, hogy `ez jó meditáció volt, ez rossz meditáció volt',
különbséget akarunk tenni és nevet adni neki.

Ilyen az elégedetlen elme. Valamivé válni akar, meg akar érkezni egy
állapotba és nevet akar magának. De sehol nem szeret megállni. Megy
tovább és tovább, amíg csak észre nem vesszük, hogy a folytonos futásban
teljesen kimerültünk.

Amikor a tudatban láthatóvá válik, hogy mi magunk tesszük ezt, a
névkeresés megáll. Megáll, mert a látás felváltotta a nem-látást, a
tudatlanságot. A tudatos látás elegendő, hogy megtörje a kényszert a
folytatásban.

A jelenben minden változik, semmi sem rögzített. Minden mozog, a
tapasztalat erre-arra fordul és folyik, nem áll meg egy fotóra és vár
amíg nevet adunk neki. Ebben a változásban, a kétséggel és aggodalommal
teli kérdések, az önazonosság és a célok elvesztik a jelentésüket. A
\emph{Mahāsatipaṭṭhāna Szutta} kifejezését használva, `\emph{szabadon
időzik, semmihez sem kötődve a világon}.'

Ennyi elég, így ismerve az elmét megállunk és megérkezünk egy helyre,
ahol hálásak tudunk lenni a létezésért. Semmilyen különös dolog miatt.
Hálásnak lenni, hogy van tapasztalat, megismerés, tisztánlátás, és a
szabadság, ami engedi, hogy megálljunk és nem kell több és több felé
mennünk.

Kiegyensúlyozott testtartásban a test kifinomult belső érzéseit könnyű
megfigyelni. Befelé irányítjuk a figyelmet, kíváncsi hozzáállással. Nem
tudjuk előre mit fogunk találni.

Ezek az érzések gyakran nem tisztán kivehetőek. Megtapasztaljuk őket, de
nincsenek tiszta határvonalaik. Nincsenek éleik, vagy határozott
formájuk. Próbálunk szavakat találni rájuk, de ezek nem illeszkednek
jól, nem vagyunk biztosak abban, hogy minek nevezzük őket.

Minden szimbólum, ami név lehetne, hiányos. A nyugati kultúránkban
erősen bízunk a tényekben, és szeretünk visszatérni ahhoz a
biztonsághoz, amit a nevekben és terminológiában érzünk. Nem vagyunk
ismerősek azzal a tudati folyamattal, ami nem használ neveket és
rögzített szimbólumokat. Az érzések, a tapasztalat maga nem tisztán
meghatározott, de tudjuk, hogy jelen van ez a tapasztalat.

Így meg tudjuk különböztetni a nevet adó folyamatot magától a
tapasztalattól. A test kifinomult érzései ködszerűek, nincsenek éles
határaik. Belégzés és kilégzés közben, megtapasztalhatjuk milyen ez az
érzés az egész testben mindenhol egyszerre. Az egész test lélegzik, van
érzés és tapasztalat, de nincsenek nevek és éles határok.

A nevet adó folyamatot elhagyjuk, és észre vesszük, hogy képesek vagyunk
ismerni ezeket az érzéseket, ahogy jelen vannak. A megismerő elme örömét
találja abban, hogy szűrők nélkül szélesebb körben fogja be a
tapasztalatot. Tudjuk, hogy milyen a tapasztalat, anélkül, hogy nevet
kellene találnunk rá.

A kártékony elme állapotok érzetében észrevehetünk egy heves érzetet,
nyugtalanságot, elégedetlenséget és szorongást. Emlékezünk, hogy
türelemmel forduljuk felé, és fenntartsuk a kitartást az állapot
érzéseinek jelenlétében. Ez is meg fog változni, ez is el fog múlni, és
meg tudjuk ezt várni. Amikor tudjuk hol állunk, a legtöbb esetben ennyi
elég. Az elme folyamatai maguktól meg fognak változni. Ha nem teszünk
tüzelőt a tűzre, az el fogja égetni amije van és magától kialszik.

Az elhatározás és ismétlés része a gyakorlásnak, de egy kifejezett cél
felé törekedésben az erőfeszítés keserűvé és fárasztóvá válik. `Ez már a
teljes felébredés? Vagy legalább egy kicsi? Mikor fog már szólni a
meditációs harang?' Ne egy állapotot keress. Az elme, ami felébredetté
akar válni, túlbonyolítja a helyzetet.

A jelen tapasztalat mindig egyszerű, az éber figyelemnek megvan a
képessége a bölcs megértésre. Ehhez térünk vissza, ez irányítja az
erőfeszítést, anélkül, hogy valami mássá akarnánk válni.

Ezt nem erőltethetjük akarattal, és nem garantálhatjuk mi fog történni,
bíznunk kell a folyamatban. Ami marad, az a jótékony elme ami érti mi
történik. Nem sürget minket a kényszer, és nem erőltetünk. A nehézség
után van terünk ahhoz, hogy megjelenjen a hála, a könnyedség hűs,
kényelmes érzésével.

A tanítóinkra nézünk fel példaként. Nem azért meditáltak, hogy elérjenek
egy különleges állapotot és azután keressenek valami más tennivalót. A
meditáció nem elkülönül, hanem beépül az életükbe. A \emph{szutták}, a
buddhista hagyomány megőrzött szövegeinek példáiban, a Tiszteletreméltó
Száriputta gyakorlata az volt, hogy elmében az üresség szemléletével
marad; a Buddha a jeltelenre irányuló koncentrációt tartotta fenn. Így
folytatták a meditációt.
