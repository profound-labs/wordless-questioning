\chapter{Pocsékul}

\section{Egy Felépített Kép}

\begin{itemize}
\tightlist
\item
  Helló, hogy érzed magad?
\item
  Pocsékul.
\end{itemize}

Egy meditáló embernek nem ezt kellene mondania, ugye? Pozitívan kellene
válaszoljon, mint például, `Remekül érzem magam, csodás napunk van!',
vagy legalább, `Én megvagyok, és te hogy vagy?' Van a fejünkben egy kép
a `meditálóról', akitől elvárjuk, hogy bizonyos módon viselkedjen és
beszéljen, és bizonyos más dolgok nem illenek hozzá. Hogy került a
fejünkbe ez a kép?

A `jó meditáló' képe egy felületes benyomások alapján kialakult
észlelés, amiről engedtük magunkat, hogy valósnak tekintsük mélyebb
vizsgálat nélkül. Gondolj arra, mikor megnyitottál egy meditációról
szóló cikket. (Miközben tovább olvasod a jelenlegit\ldots) Egy
fényképpel kezdődik amin egy szerzetes vagy meditációs világi tanító
mosolyog, és az éberség pozitív hatásairól szóló leírással folytatódik.
Esetleg egy elvonuláson készült fényképeket és történeteket is
tartalmaz. Az emberek békés arccal ülnek a meditációs párnákon, míg az
ablakon keresztül áradó fény megvilágítja a Buddha szobrot. Lentebb egy
interjú részlet olvasható benne arról, hogy valaki hogyan jutott túl a
belső küzdelmein. Egy meditációs tanító bátorító szavaival zárul a
tiszta elme erejéről, vagy egy idézettel a Buddhától.

Még akit nem is érdekel túlságosan a meditáció is tudja milyen egy ilyen
cikk, mind láttunk már több tucatot. Nincs szükség itt forrás szöveget
megjelölnöm, ebben a könyvben található fejezetek is például
szolgálhatnak.

Ezzel nem arra akarok utalni, hogy félre akarnának vezetni. A szerzők jó
szándékkal teszik ezt, hogy bátorítsanak minket a vizsgálódás ösvényén
haladni és hogy tegyünk erőfeszítést a gyakorlásunkban. Ha a
frusztrációkon túl nem látható semmilyen nagyobb boldogság, mi az
értelme az egésznek? Ha csak kín és szenvedés lenne várható, azt
segítség nélkül is létre tudjuk hozni magunknak.

A buddhizmus alapvetően optimista, és központi témája a boldogság. Saját
hozzáállásunk, hogy a boldogságot keressük, vagy létre akarjuk hozni.
Míg ezzel foglaljuk el magunkat, csak egyre keserűbbek leszünk, és úgy
tűnik ez sosem lesz megvalósítható. Emiatt a körülményeket, vagy saját
képességünket tartjuk elégtelennek, de valójában a dolgok természetes
működését nem értjük, és hozzáállásunk ezért vezet helytelen irányba. A
feladatunk, nem az, hogy a boldogságot keressük, hanem, hogy a szenvedés
keletkezését megértsük, és ne hozzuk azt létre. Ennek megértésén
keresztül az elmében spontán megjelenik a boldogság.

A tanítások valóban sokat említik a szenvedést, de az utasítások arra
alapulnak, hogy a szabadulás ettől a szenvedéstől lehetséges. A Buddha
világosan kifejtette, hogy az út gyümölcse őszinte boldogság, és ha ezt
gyakorolni nem lenne lehetséges, vagyis elhagyni a kártékony gyökereket
és fejleszteni a jótékonyakat, nem tanítaná azt.\footnote{\href{https://suttacentral.net/an2.11-20/en/thanissaro}{AN
  2.19}}

Ki ez a meditáló a fejünkben? Egy dolog biztos, aki az \emph{én
fejemben} van, mindig \emph{jobban tud meditálni} mint én. Amikor valaki
egy jó fotót készít rólunk, tudjuk hogy van ez: egy rendezett, jó
pillanat volt, de tudjuk, hogy öt percen belül akár az ellenkezője is
igaz lehet. Habár ez másokra is érvényes, nem csak magunkra, valahogy
nem emlékszünk erre amikor másokról készül jó fotókat látunk.

\section{Szálak}

A Buddha idejében az irodalom egyik formája a \emph{szutta} volt, ami
egy párbeszéd szálat jelent. Ez tartalmazhat prózai és verses szöveget
is, azzal a szándékkal, hogy szavaláson keresztül lehessen őket
memorizálni. Egy spontán mindennapos eseményt, vagy szervezett közösségi
gyűlést követően ami tisztán érthető példája volt a tanításoknak, a
közösség formális alakba öntötte a történetet és szuttaként szóban
memorizálták azt. Erre a Buddha is szorgalmazta őket:

\begin{quote}
Így kell gyakorolnotok: `Jól oda fogunk figyelni, amikor a
tanítóbeszédeket recitálják, amelyek a Tathágata mélységes, mély
jelentésű, páratlan, az ürességre vonatkozó szavai. Hegyezni fogjuk a
fülünket, kitárjuk a szívünket a megismerésük felé. Úgy tekintünk rájuk,
mint amit megéri megragadni és a mesterévé válni.'
(\href{https://a-buddha-ujja.hu/sn-20.7/hu/fenyvesi-robert}{SN 20.7})
\end{quote}

Manapság a könyvek, cikkek, blog posztok töltik be ezt a szerepet,
melyek a kánon szuttáit tartják maguk előtt példaként. A korai idők
szerzetesei ezeken a szálakon keresztül juttatták el hozzánk üzenetüket,
ez részévé válik az egymás közötti párbeszédünknek azokkal, akikkel ma
találkozunk, és írott munkáink azokhoz szólnak a jövőben, akikkel mi már
nem találkozunk.

Viszont a mai szociális médiában az üzenet világos megértését torzítja
az úgynevezett `Instagram effektus', ami egy szelekciós elfogultság
afelé, hogy a legjobb és legpozitívabb oldalunkat mutassuk, és szűrjük
ki a negatívat, ami ennek ellenére éppen olyan valóságos és szükséges az
egész kép teljes megértéséhez.

Gondold végig az írás folyamatát. A szerzőnek van egy mondandója, ami
egy cél lesz amihez igazodjunk, vagy egy tapasztalat amit felismerjünk.
Kiválaszt egy sor személyes tapasztalatot, véleményt, és megerősítő
magyarázatot más szerzőktől.

A szerző írhat igazság-hűen és próbálhatja elkerülni az `Instagram
effektus' szelekciós elfogultságát, de valamilyen szűrő mindig
működésben van. A saját tapasztalatából választ és bemutat egy szeletet,
lefordítva azt az olvasók kultúrájának kifejezéseire, olyan példákra
építve, amik (remélhetőleg) ismerősek azon olvasóknak. Az írott szó
világa mindig egy felépített valóság. Viszont, ha jól el van találva,
ráismerünk benne a saját tapasztalatunkra.

A meditáló, aki a fejünkben él olyan, mint egy versben szereplő
karakter, vagy egy mítoszban szereplő hős. Bölcsebbek és erősebbek
lehetnek nálunk, hogy amikor elveszettnek és gyengének érezzük magunkat,
hitet és tanácsot tudnak nekünk adni. Az ő békéjük megrendíthetetlen
lehet, hogy amikor pocsékul érezzük magunkat, képesek legyünk kiállni
azt és várni amíg a nehézség véget nem ér.

Az ilyen felépített mentális képeket nem tekinthetjük valódi személynek.
Értékes források amikkel önmagunknak irányt mutathatunk, a történet
leírása segít kitalálnunk mit tegyünk azzal, hogy megmutatja hol vagyunk
egy nagyobb kép keretén belül. Egy mentális kép szerepe nem az, hogy
\emph{mivé kellene váljunk}. Amikor ezt próbáljuk elérni,
önellentmondásokba keveredünk és elégtelennek érezzük magunkat, mert az
élet valós körülményei sokkal összetettebbek, képlékeny és változó
határokkal, nem mint egy kép leegyszerűsített valósága. A kép egy eszköz
a magyarázatra, egy adott világra tekintő \emph{látásmód}, és egy példa
a helyes cselekvés módjára abban a fajta világban.

\section{Feltevések}

Felidézhetjük a Dhammapada verset, ami rámutat, hogy a tapasztalatunk
világa nem független tőlünk:

\begin{quote}
Az elme minden létállapot előtt jár, az elme vezeti őket, az elméből
származnak. (Dhp 1)
\end{quote}

Hogyan tesztelhetjük, hogy képzeletbeli problémákat gyártunk magunknak,
vagy a tettre késztető nyomás egy fontos jel, amit nem kellene figyelmen
kívül hagynunk?

Először megkérdezhetjük, `Képes az alany szenvedést tapasztalni?'
Élőlények szenvedhetnek, de egy kulturális fogalom, vagy magunk által
létrehozott történet nem tud szenvedni, még ha közben \emph{mi}
szenvedünk is. Megváltoztatja a hozzáállásunkat, ha az aggodalmunk
tárgya csak történetként létezik, mint egy intézmény, nemzet, pénz,
hírnév vagy egyéb társadalmi történet, és nem egy élő lény.

Ezt követheti egy gyors morális biztonsági teszt: `Egy bölcs ember vajon
dicsérné vagy kritizálná ezt?'

Majd visszaemlékezve kibogozhatjuk a nézetet: `Milyen feltevés hozza
létre ezt a feszültséget és nyomást? Mi ad jelentést nekem ahhoz, hogy
ezt tegyem? Mi az, ami nélkül ennek nem lenne jelentősége?'

Feltárhatjuk az ilyen tudattalan motivációkat azzal, hogy a jelen
tetteinket és választásainkat figyeljük. Ami már kifejezésre került,
implicit tartalmazza az információt, ami a megjelenéséhez szükséges
volt. `Miért választom megtenni ezt, itt? Honnan ered ez a tett és hova
vezet?' A mögöttes tényezők eredhetnek például a környezetünk által
kondicionált szokásokból, és eddig sosem fejeztük ki gondolatban miért
tesszük amit teszünk, csak a tett eredményét éreztük magunkon, legyen az
jó vagy rossz.

A tettekkel kezdeni a vizsgálatot és úgy rákérdezni a gondolatokra egy
megbízhatóbb módszernek bizonyulhat, mert a belső csevegésünk közben
mindenféle belső ellentmondásokat mondunk magunknak, viszont egy tett
már ad egy világos referencia pontot.

A hozzá kapcsolódó érzés lehet, hogy pocsék, de ha ezt jelzésként
kezeljük arra, hogy forduljunk az elme felé és vizsgáljuk azt, a
hozzáállásunk gyakorlatias és eredményes maradhat. `Ha már egyszer itt
vagyok, mit tanulhatok ebből?'

A feltevéseinkhez azon keresztül találunk hozzáférést, hogy felfedjük a
tudattalan motivációinkat -- ha egyszer már tisztán ki tudunk fejezni
egy feltevést, szabadságot nyerünk arra, hogy megfordítsuk, vagy
elhagyjuk azt. Megkérdezhetjük, `Segít ebben a helyzetben, ha
megfordítom a feltevéseimet?' Talán az, hogy az ellenkező irányból
tekintünk rá lesz éppen az, amire szükségünk volt a megbékélésre, a
tisztánlátásra a folytatáshoz, vagy ahhoz, hogy elhagyjuk az egész ügyet
mint ami sosem létezett. Akárhogy is, már nem kényszerből cselekszünk:
szabadok vagyunk elengedni vagy \emph{választani} azt, hogy végig
folytatjuk.

\section{A Vihar Után}

Amikor az utasítás azt mondja, `tér vissza a jelen pillanathoz', ez nem
jelenti, hogy mindent szeretned kell amit ott találsz, de ez az egyetlen
hely ahol élni tudsz. Ha boldog vagy, nem a jövőben vagy boldog, hanem a
jelenben. Ha szenvedsz, nem értheted meg a jövőben, csak a jelenben.
Egyes helyzeteket semmilyen agyalás és belső párbeszéd nem fog javítani,
legjobb úgy nevezni ahogy az van, és méltósággal eltűrni. Egy konfliktus
őszintén feszültséggel teli, elválni attól amit szeretünk mindig szomorú
lesz, és életben lenni mindig a saját halálunk tragédiájával végződik.

Hajlamosak vagyunk a sikert várni, és számítunk arra, hogy a kemény
munkánk a jövőben igazolódik. Vedd szemügyre óvatosan a siker
pillanatát, mit tapasztalsz? Lehet meglepetés, öröm, vidámság,
megkönnyebbülés, az után minden visszatér a hétköznapi szintre. A célról
kiderül, hogy nem az, aminek gondoltunk, és ha intenzíven arra
koncentráltunk, hogy oda jussunk, talán nem is emlékszünk semmire az
odavezető útról, és bánkódunk, hogy elmulasztottuk. Olyan erősen leköt
minket az, hogy eredményesek legyünk, hogy elpazaroljuk a lehetőségünket
arra, hogy éljünk.

A halál felett szemlélődni egy igazmondó, noha kissé ilyesztő, tükröt
tart az értékeink elé. `Ha ma este meghalok, boldogan emlékeznék arra,
hogy úgy élek, ahogy ma teszem?' Ez a kérdés többet fel tud kavarni a
mélyből, mint amire vállalkoztunk. Emlékszem olyan időre, amikor a
válaszom a `boldog' szóra kizárólag harag és önutálat volt.

A `hedonikus taposómalom' kifejezés leírja az adaptív folyamatot, amiben
a sikeres eredmények érzelmi hatása a normális szintre csökken az
elmében, és visszaáll a szokásos érzelmi állapotunk. Mintha
taposómalomban járnánk, nem számít milyen erősen próbálja az ember
növelni a boldogság szintét sikerek elérésével, az ember csak egy
helyben marad. Az életünket azzal töltjük, hogy az úton utazunk, nem a
célállomásban. Ha közelebbről megnézzük, még a célállomás puszta ötlete
is szétfoszlik, minta mikor berepülünk egy felhőbe. `Azt hittem ott
látom, de most, hogy ott vagyok, itt semmi sincs.'

Az bölcsek figyelmeztetnek erre minket, de úgy látszik szenvednünk kell
mielőtt figyelni kezdünk, hogy megértsük mi az a probléma amiről
beszélnek. Szókratész azt mondja, `Óvakodj a nyüzsgő élet
terméketlenségétől', de abból amit az életéről tudunk, nekem úgy tűnik
egy aktív és törekvő ember volt. Talán ez nem ellentmondás, ha az ember
örömét a jelenben találja, és nem a jövőre néző elvárásokban keresi. Azt
is mondja, `A boldogság titka nem a több keresésében rejlik, hanem
abban, hogy képesek vagyunk élvezni a kevesebbet.'

Könnyen túl-korrigáljuk a nyüzsgést, és átesünk a másik végletbe, azt
gondolva, `Elegem van! Megszabadulok mindentől!' Az önutálat
``logikusnak'' tűnhet, még ha tovább folytatja is a szenvedést. Sokan
vagyunk, akik könnyen kritizáljuk magunkat, és szorgalmasan gyakoroljuk
ezt, olyan meggyőződéssel igyekszünk bebizonyítani a saját tévedésünket,
mintha az ön-ellenszenv egy erény lenne. `Pocsékul érzem magam, aki
\emph{valóban} meditál sosem érezné így magát. Biztos, hogy valamit
rosszul csinálok.' Egy egész ön-azonosságot fel lehet építeni ekörül,
ami mindig panaszokkal és ön-ellenszenvvel válaszol. Egy adott személy
évtizedeken át élhet így, és ez válik az alap szintté, ami alapján
felismerik magukat. `Ha nem lennék ilyen mérges, nem is tudnám ki
vagyok.'

Olyan ez, mint beragadni egy tükrökből készült labirintusba: bárhova
nézel, csak magadat látod. A menekülés kulcsa, hogy találjunk egy
repedést a tükrökön és ismerjük fel a változást: az hajtott érzés,
szorongás, harag, motivációk amikről azt gondoltuk állandóak, valójában
folyton változnak, szétesnek és újra formálódnak. A labirintust az elme
hozta létre, és amit létrehozott üres az éntől, nem lehet az, ami
valójában mi vagyunk.

Tudunk benne meggyőző logikát találni, és az érvelésünk a kritikus
hozzáállásunk védelmében teljesen észszerű lehet! A pszichológusok azt
mondják, hogy a legnehezebben kezelhető betegeik azok, akik
intelligensen védik és indokolják saját rossz szokásaikat. Emlékszel
magadra, amikor te játszottad a keserű filozófus szerepét? Olyan okosak
tudunk lenni, semmi esély arra, hogy boldogok legyünk, és be is tudjuk
bizonyítani.

Nem szükségszerűen jelent azonnali megkönnyebbülést, amikor
ön-vizsgálatunk felfedi nekünk az eddig keresett értékeink ürességét. A
harag, kétségbeesés\footnote{A Buddha a haraggal és kétségbeeséssel való
  küszködést ahhoz hasonlítja, mintha egy ösvényt követnénk, ami mellett
  mély szakadék tátong.
  (\href{https://www.accesstoinsight.org/tipitaka/sn/sn22/sn22.084.than.html}{SN
  22.84})

  A türelmes kitartás egy alábecsült erény, de gyakran nincs másra
  szükségünk, csak hogy eszünkbe jusson várni: a kavargó elme állapotok
  drámai eső és villámlása ki fogja magát futni. Amikor megjelenik a
  hála érzete, az olyan jel, mint a vihar utáni szivárvány. Jótékony
  elme állapotokat kísér, és intelligensen több szögből is látjuk a
  helyzetet: egy jó alap arra, hogy segítőkész gondolatok jelenjenek meg
  arról, mit tegyünk. Néha az a legjobb, ha egyszerűsítünk és
  elfordulunk régi szokások és értékektől, Máskor, már megváltozott a
  nézetünk, és talán tovább folytatjuk amivel eddig foglalkoztunk, de
  hátra hagyva a nagy sietséget, azért folytatjuk, hogy azt éljük, nem
  mert a jövőt várjuk.

  \begin{quote}
  A múltat ne kergesd,\\
  és ne álmodozz a jövőről.\\
  Ami elmúlt az már mögöttünk van.\\
  Ami eljön azt még nem értük el.

  \emph{MN 131, Bhaddekaratta Sutta}
  \end{quote}} és szomorúság gyakran megjelennek, mint az első reakció,
és önutálattal foglalkozó gondolatokat generálnak. Ezek az elme
állapotok nem megbízhatóak, de szerencsére ezek is üresek az éntől, nem
én és enyémek, és ráadásul lezárják az intelligenciánkat, és az kinek
kell?

\section{Humor és Irónia}

A mogorva, sötét hangulatok olyanok, mintha a saját logikai csapdánk
zárult volna a lábunkra, és hiába erőlködünk, nem tudjuk lerázni. Minél
többet gondolkodunk róla, a zár annál szorosabb lesz. A humor és irónia
éppen azért vicces, mert váratlan, furcsa szögből mutatják a helyzetet.
Ha a széles út egyenesen előre el van zárva, miért ne próbáljuk meg az
oldalcsapást ahol a róka jár? Egy vicc nem lenne vicces ha logikus és
észszerű lenne -- így a humor és irónia, önmagunk felé irányítva, jó
barátnak bizonyulnak mikor a szokásos észszerű gondolataink csak egyre
több észszerű szenvedést gyártanak.

Mitől lesz az öreg és bölcs ember \emph{bölcs}? Orvosi
tanulmányok\footnote{\href{https://www.researchgate.net/publication/258190619_Aging_irony_and_wisdom_On_the_narrative_psychology_of_later_life}{\emph{Aging,
  irony, and wisdom}, William Randall, 2013}} megvizsgálták az idős
emberek különféle szemléletmódjait, és azt találták, hogy a hajlamosság
az önmaguk felé irányított humorra és iróniára (vagyis amikor az ember
képes nevetni önmagán) lehetővé teszi számukra, hogy szembenézzenek az
öregedés jelentős kihívásaival, miközben megőrzik szellemi egyensúlyukat
és pozitív hozzáállásukat az élethez.

Egyik központi megfigyelésük az, hogy a humor és irónia fejleszti a
képességünket abban, hogy önmagunkat többféle nézőpontból is lássuk.
Egyidejűleg betölthetjük a pontos történész és a tréfáló komédiás
szerepét. Így többféle narrátori szögből is tudjuk látni az eseményeket,
és mivel nem ragadunk be egyetlen történetbe, a narrátori keret amiben
magunkat látjuk, nyitott marad, és egy pozitív jövő irányába halad. A
létezésünk korlátai nem szükségszerűen jelentik a történet végét, és a
nevetést nem nehéz megtalálni: az élet abszurd sarkairól mindig lehet
egy jó viccet mondani.

Talán érzéketlen dolog valaki más rossz helyzetéről viccelődni, de ki
fog felháborodni a magadról szóló humoros megjegyzéseidről? Ha pocsékul
érzed magad, mit szólsz egy pocsék vicchez? Ez a menet olyan rossz hogy
már jó, és a jegyek ingyen vannak. `Egy életre kelt csontváz vagyok egy
bőrzsákban amire ruhákat aggatok, mesés frizurám alatt a \emph{fontos
véleményeim} logikáját bizonyítgatom.' Hol nincs ezen nevetnivaló?

Gyakran mondjuk, hogy meditáció közben megfigyeljük a mentális
szokásainkat, de néha ezt egy kritikus elfogultságával gyakoroljuk,
megfigyeljük a \emph{rossz mentális szokásainkat}, és nem vesszük észre
a jókat. Annyira jók tudunk lenni abban, hogy figyelmen kívül hagyjuk a
kellemes elme állapotokat, hogy az ember őszintén elhiszi, hogy a
boldogság csak mások számára létezik. Amikor valami jó történik és
boldognak érzed magad, állj meg és vedd észre, `Na, ez milyen jó.' Ki
fogja észre venni, ha te nem? Képezd az agyat, hogy aktívabban reagáljon
rá: ha valami jót tettél, túljutottál egy akadályon, befejeztél egy
feladatod, jutalmazd meg magad.

Ne becsüld alá a kicsi, de helyben elérhető és rendszeres szokásokat. A
nagy célokat a jövőben nehéz konkrét tettekre lefordítani, de gondolj
arra, mi lenne egyetlen kis jó lépés amit most is meg lehet tenni?
Amikor kész, mentálisan vagy szóban ismerd el és dicsérd meg magad. Az
agy erre úgy válaszol, hogy a dopamin nevű neurotranszmittert termeli,
aminek felvillanyozó hatása van az általános közérzetünkre, így magunk
formálhatjuk annak dopamin-válasz ciklusát. Az agy megtanulja jutalmazni
a nehéz munkát, és nekünk is aktív részünk van abban, hogy arra
képezzük, mi is az, amit ki kellene emeljen értékes tapasztalatként. Ha
az agy nem kap jutalmat a nehéz munkáért, nem tanulja meg, hogy dopamint
termelve jó érzéssel töltsön el, és így csak küszködés lesz, jó érzés
nélkül.

\section{Elvárások}

Az ember ránéz egy Buddha szoborra, és talán azt várja el magától, hogy
hasonlóan tökéletes testtartással meditáljon egyetlen mozdulat nélkül,
akár csak a Buddha. Ebben az esetben viszont a külső jelekre vonatkozó
elvárásaink félrevezettek minket, és nem vettük észre a valódi üzenetet,
ami belső tulajdonságokra mutat. A Buddha szobrok nem a Buddhát
ábrázolják, a történelmi \emph{Sziddhárta Gótamát} aki az i.e. 5.
században élt. A szuttákban kifejezetten utasította a szerzeteseket,
hogy \emph{ne} készítsenek szobrokat vagy képeket róla, és ehelyett a
Dhammára, az elme igazságaira fordítsanak figyelmet. Az első Buddha
szobrokat négy vagy ötszáz évvel a halála után készítették a görögök, az
afganisztáni Gandhára régióban. A szobrok a felébredett elme
bölcsességét és nyugalmát jelképezik, ahogy az emberi formában ez
kifejeződik.

Gyönyörű rájuk nézni, de senki nem fog Buddha szoborrá válni, mint ahogy
nem válhatsz a tökéletes meditáló képévé sem, vagy a hőssé egy lírikus
költeményben. Tanácsot valóban adnak, de a tanács nem tud irányba
igazítani, ha mereven értelmezzük és nem használjuk az
intelligenciánkat. Úgy kell alkalmaznunk, hogy figyelembe vesszük a
belső tapasztalatunkat és jelen helyzetünket -- így visszatérünk a
tudathoz, ami ráébred az igazságra és túllép az akadályon. Az erény
gyakorlása és a bizalom a nagy képességű tanítók példájában erős alapot
képez, ami ott van, hogy magunknak jót kívánhassunk és mégis el tudjuk
ismerni, hogy pocsékul érezzük magunkat, ha éppen olyan a helyzet.

Az elvárások előrejelzik egy eredmény elvárt értékét, előre megbecslik a
helyzetünk kimenetelét. Eközben, minden tényező ami beszámít az
előrejelzésbe folyamatos változáson megy át -- így engednünk kell az
előrejelzést is változni, elvárásainknak a mentális tapasztalatunkról
folyamatosan változniuk kell aszerint, hol állunk éppen most. Ha
természetüknek megfelelően felismerjük őket, nem jelentenek problémát,
de ha ragaszkodunk egy bizonyos változathoz amit `az igazinak' hiszünk,
éppen az válik akadállyá. Amíg ragaszkodunk a szomjas vágyhoz, az
kényszerít minket, nem vagyunk képesek megállítani, és folyamatosan
szenvedést okoz nekünk. Végül kiderül, hogy ha a jövőbeli érzelmi
állapotunkba fektetjük a boldogságunk lényegét, az eredmény többnyire a
csalódás.

A légzés meditáció technikáját a Buddha 16 lépésben tanította. Mi az
utolsó lépés? Mi lehet az a fenséges elme állapot, amit végül magunkénak
tudhatunk? Az utasítás rávezet, hogy nem ez a helyes hozzáállás. A
légzés meditáció a test, az érzések, az elme állapot szemlélete után a
természetes igazságok szemléletét tanítja, melyek utolsó lépése:

\begin{quote}
`Az elhagyás fölött szemlélődve lélegzem be, így gyakorol. Az elhagyás
fölött szemlélődve lélegzem ki, így gyakorol.' (MN 118)
\end{quote}

A Nemes Nyolcrétű Ösvény gyakorlása nem a halmozásról szól, hanem az
értékeink átalakulásáról a megfigyelésen keresztül, a körülmények
változásának tapasztalatát éberen vizsgálva. Végül elhagyjuk őket,
mintha letennénk egy terhet, nem cipeljük azt tovább. Ebbe minden
beletartozik, amit az `én és enyém' magába foglal: meddig tudunk bármit
is megtartani? A halandóságunk fölött szemlélődve, az én létezésének
korlátozott ideje felett, olyan értékek felé vezet, melyek túllépnek
magunkon. Az ilyen, egót meghaladó nézőpontból erednek a kedvesség,
kiegyensúlyozottság és boldogság jellemzői. Ezeket nem az akarat hozza
létre, hanem spontán megjelenő állapotai az elmének, ami szabad a
szomjas vágyakozástól.

FIXME: elhagyás nem tétlenség, erény

A vizsgálódás és fejlődés utat enged a változásnak, és megnyit egy tágas
látószöget, amiben az ellentétek együtt tudnak létezni összetett
kapcsolatokban. Egy ellenkező megközelítésben, a bíráló és ítélkező elme
korlátozott körben mozog, és minden dolgot rendszerezni akar szabályos,
egymást kölcsönösen kizáró absztrakt kategóriákba. Ez utóbbi hozzáállás
könnyen bizalomvesztéshez és ártalomhoz vezet -- elkezdünk hitet
veszíteni, nem hisszük magunkról, hogy `valódi' gyakorlók vagyunk, és
egyúttal mások sem tűnnek hitelesnek. Nem csak mi magunk nem tudunk
tanulni, de senkit sem tudunk elfogadni, hogy tanítson minket. Ez a
kétség megvakít és megbénít, minden erőfeszítésünk megáll.

Az mondani magunknak, hogy a fájdalom nem fájdalmas, nem olyan
meditációs technika amit a Buddha tanított. Mi találjuk ki, mint egy
fedő történetet, amikor úgy gondoljuk, hogy olyannak kell lennünk mint a
mitológiai ideáloknak. A meditáció nem egy erő, amivel irányítani tudjuk
a mentális állapotokat, hanem a tudatosság fejlesztése arra, hogy
megálljunk és azok ne irányítsanak minket.

\section{Kalibrálás}

Egyes orvosi tanulmányok arról számolnak be, hogy az érzelmek hasonlóak
egy előrejelzéshez.\footnote{\emph{How Emotions are Made} by Lisa
  Feldman Barrett} Az agyunk kiértékeli a jelent a múlt alapján, és
annak megfelelően reagál, hogy ez alapján jó vagy rossz várható. Az agy
folyamatosan kapja a jeleket az idegrendszertől, és az alaján, hogy mit
tanult a múltbeli tapasztalatok alapján, próbálja megítélni, hogy vajon
a jelen helyzet energia bevitelt vagy energia kiadást fog jelenteni a
test számára. A választ hormonok formájában fogalmazza meg, melyek
létrehozzák a testi reackiót, mint amilyen a veszélytől való félelem, az
azonnal várható jutalom izgalma, vagy a szárnyaló boldogság.

Hajlamosak vagyunk azt hinni, hogy a tapasztalatunk olyan, mint a
látványt ami egy ablakon át láthatunk: ott állunk előtte, és habár az
érzékeink nem is egészen élesek, befogadjuk az elénk táruló látványt, és
egy többé-kevésbé teljes képünk van arról, `mi is van odakint'.

Gyorsan kiderül, hogy ez a kép inkább kevésé teljes, mint többé, ha
meggonduljuk hogyank működnek az érzékek és az idegrendszer. Az agy nem
kap túl sok információt, amivel dolgozia kell, és néhány egyszerű jelből
meg kell tippelnie, milyen lehet a gazdag világ, rajta kívül van.

Nem lát túl sokat a külső világból: ott kuksol a koponyában, ami egy
sötét doboz. Testi folyadékok, vegyületek és idegrendszeri jelek
üzeneteket visznek ebbe a dobozba. Az üzenetek a test már rendszereitől
erednek, amik maguk is zajosak és néha egymástnak ellentmondanak. Ebből
a kavalkádból az agynak létre kell hoznia az észlelést arról, hogy hol
vagyunk, megtippelnie mi történik velünk, megjósolnia valószínűleg mi
fog történni a következő pár percben, és produkálnia kell egy választ,
ami remélhetőleg segít bennünket a túlélésben, vagy akár még a
boldogsághoz is vezethet. Ezt mind ell kell végeznie, egy sötét dobozon
belülről, néhány zajos és korlátol jelzés alapján.

Mi vagyok tehát, egy életre kelt csontváz, a fejem pedig egy sötét
doboz? Ez sok zavarodottságot megmagyaráz. Csoda-e, hogy az elvárásaim
egy kicsit félrecsúsznak, és folyamatos igazgatásra van szükségük? Amit
valóságként tapasztalok, egy folyamatban lévő találgatás eredménye, ami
másodpercenként változik.

`A boldogság egyenlő a valósággal, mínusz az elvárások' -- Tom Magliozzi
könnyen megjegyezhető mondása. Manapság az elvárásaink olyan magasak.
Naponta többször frissítéseket kapunk a szociális média appoktól, web
cikkeket olvasunk, és minden alkalommal befolyásolják a nézetünket
arról, hogyan vagyunk, és hogyan áll a világ körülöttünk. Lehetetlenül
tökéletes, lehetetlenül elhatározott, lehetetlenül felháborító képeket
mutatnak nekünk más emberekről, megszűrt és precízre vágott képeket a
meggyőző benyomás érdekében. Mivel nem találkozunk ezekkel az emberekkel
szemtől-szembe, nem látjuk az életük valóságos hátterét, és ez
felnagyítja az elvárásainkat. Újra és újra arra képezi az agyat, hogy
ezeket a mesterségesen létrehozott benyomásokat várja el, mint egy
túlhajtott elvárás-gép. Nem is vesszük észre ezt a torzított
ön-kondicionálást, már csak a csalódást és kimerültséget, amit a
szüntelen elégedetlenség okoz.

Viszont \emph{megvan a képességünk}, hogy kalibráljuk ezt az `elvárás
gépet', a tudatos vizsgálódás és töprengés kiegyensúlyozó hatásával.
`Mik a mai nap leglényegesebb felelősségei, amiről gondoskodnunk kell?
Testileg, mire van szükségünk egy naphoz?' Ha leegyszerűsíted a választ
a lényegre, nem olyan sok marad. Étel, ruha, szállás, gyógyszer,
támogató szellemű társak és talán valami tennivaló egy érdemes cél
irányába. Az átlagos nap persze nem fog igazodni egy ilyen absztrakt,
tiszta egyszerűséghez, de ez arra szolgál, hogy felismerjük az alapvető
szintet. Ha az egyszerű is elegendő, akkor nem jelent problémát, hogy
többet is tudunk tenni, vagy több mindenhez is hozzáférünk, a
megelégedettség marad az alapvető szintünk. A törekvés nem probléma, de
az elvárások felnagyítása blokkolja annak megvalósítását.

Az elvárások szükségesek ahhoz, hogy egy adott irányt kövessünk a
világban, de ha nem értjük őket, akadályokká válnak a szívben. Az
elvárások és érzelmek természete hogy megjelenjenek, ide-oda fordulva
változzanak, majd engednünk kell hogy tovább ússzanak, mint falevelek a
csónak mellett. A rosszak nem olyan rosszak, a jók nem olyan biztosak.
Ismerve a változó természetüket, nem vesszük őket olyan komolyan, és nem
akadunk fenn bennük, mint ahogy egy csónaknak sem kellene fenn akadnia
holmi száraz leveleken.

\begin{quote}
Legyen kellemes vagy fájdalmas,\\
a semlegessel együtt,\\
Akár belső, akár külső,\\
Bármilyen érzés ami van:\\
Megismeri, `Ez is szenvedésnek van kitéve,\\
megtévesztő és szétbomló',\\
Újra és újra érintve őket,\\
elmúlásukat szemlélve,\\
a szenvedélytől megszabadul.

\href{https://suttacentral.net/sn36.2/pli/ms}{SN 36.2}
\end{quote}

\section{Virágzó Élet}

A modern nyugati kultúránk a boldogságot gyakran úgy mutatja be, mint
egy meghatározott érzés, vagy egy bizonyos élethelyzet ahova el kellene
érkezzünk. A kultúrát az egymással való párbeszéd adja át, és a
boldogságról olyan módon beszélünk, mintha az egy eredmény lenne, egy
esemény a jövőben, vagy egy jól meghatározott létállapot. Úgy tűnik, ez
egy nemrég kialakult irányzat, de nem egy kifejezetten jótékony hatású.

Hagyomány szerint az ókori görögökre úgy tekintünk, mint akiknek a
legnagyobb befolyásuk volt a nyugati értékeink kialakulásában, és mások
mellett Arisztotelészre\footnote{384-322 BC}, mint aki közöttük is
kiemelkedik. Ebben nagy szerepe van, hogy sok írása fennmaradt számunkra
most is elérhető formában, és ezekben a boldogság kérdését részletesen
vizsgálja. Láthatóan erősen foglalkoztatta, hogy mi is ez, és hogyan
élhet az ember boldogan, viszont a boldogság tapasztalatát ő más
szemszögből látta.

A görög szó, amivel a `boldogságra' utal az \emph{eudaimonia},
fordításban `emberi jólét, virágzó élet.' Úgy látta ezt megjelenni, mint
egy állandóan aktív folyamatot, amit nap mint nap gyakorlunk, nem pedig
egy eredményt, ami egyszer majd a jövőben megjelenik. A boldogság
gyakorlását a morális erényekre alapozza, és az ember saját életére
vonatkozó valósághű szemléletre, ami a születéssel, a növekedés éveivel,
és az öregkorral együtt magába foglalja az ember saját halálának
tragédiáját.

Az erény és halandóság ilyen közvetlen szemlélete sorba rendezi előttünk
a dolgokat: olyan tág nézőpontot ad, amiben a boldogság a jótékony
tényezők alapjára épül, de végső soron önmagunkon túl kell néznünk
ahhoz, hogy jelentést adjunk annak. Az elvárásainkat így képezve, a
boldogság gyakorlata minden nap teljes egész. Megtanulunk a nehézséggel
együtt lenni, ha éppen úgy áll a helyzet, és legjobb képességünket
erényesen alkalmazva minden nap végén megnyugvással tekinthetünk vissza.

Emlékezhetünk, hogy magunknak jólétet és boldogságot kívánunk,
családunknak és barátainknak is boldogságot kívánunk az életükben. A
szellemi kitartást és önbecsülést úgy építjük, hogy tudatosan felidézzük
a morális erényeket, akár másokban, mint tanítók és példaképek esetében,
vagy a magunk tapasztalatában. Ez fejleszti az örömöt és értékelést,
amit mások sikerei és jósága nyomán érzünk, osztozunk a sikereikben.

Nem szükséges, hogy a szociális média appokon idegenektől várjuk, hogy
elismerjenek minket. Sokkal gyümölcsözőbb a szemtől-szembeni
kapcsolatokat fejlesztenünk olyan barátokkal, akik őszintén örülnek a mi
erőfeszítéseink sikerének. A humorral feloldhatjuk a mogorva
hangulatunkat, és megtesszük a következő lépést, ami előre visz.

A jelen maga a változás. Ezt a tapasztalatot éberen figyelve vizsgáljuk
a testet, az érzéseket, elme állapotokat és a dolgok természetes
igazságát a \emph{Szatipatthána} szutta\footnote{\href{https://a-buddha-ujja.hu/mn-10/hu/toth-zsuzsanna}{MN
  10}} refrénjét követve:

\begin{quote}
\ldots{} Úgy időzik, hogy a keletkezés természetét szemléli, vagy úgy
időzik, hogy az elmúlás természetét szemléli, vagy úgy időzik, hogy a
keletkezés és az elmúlás természetét szemléli. \ldots{} Szabadon időzik,
semmihez sem kötődve a világon.
\end{quote}
