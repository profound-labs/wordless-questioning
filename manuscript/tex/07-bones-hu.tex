\hypertarget{csontok-1}{%
\chapter{Csontok}\label{csontok-1}}

Watching the body\ldots{}

A testi tudatosság megállítja a haragot, neheztelést és belső
vádaskodást. Gondolkodva csak jobban beássuk magunkat, de ezt ilyenkor
nem látjuk. Valahogy fontosnak érezzük, hogy mérgesek legyünk, pedig
csak az idő telik el vele fájdalmasan. Arra vágyunk, hogy vége legyen és
nyugodtan mehessenek a dolgok tovább.

Megfigyeljük a test részeit, és látjuk, hogy nem hordoznak magukkal
semmilyen sztorit. Így fellélegezhetünk, hogy nem vagyunk a sztorikhoz
láncolva, azokat mi hozzuk létre.

Ehhez a figyelemhez mindig vissza tudunk térni, egy belégzés és kilégzés
elég ahhoz, hogy emlékezzünk a keletekzésre és elmúlására, és a
problémáink olyanná válnak, mint a sztorik egy régi újságban. Ráununk
kibogozni a szálakat, mintha valaki más álmait kellene értelmeznünk. Ami
a valóság, az mindig itt van a jelen tapasztalatunkban. Nem az válik
fontossá, hogy mi a sztori, hanem az, hogy a figyelmünket annak tudjuk
szentelni, ahol most vagyunk.

A testi tudatosság enged a kívánságokból és rávezet arra, hogy
szerencsések vagyunk, hogy itt lehetünk.

Hova akarunk jutni? Elkezdhetjük most. Ha valóban érdemes dologról van
szó, szinte biztos, hogy nehéz is. Ha nehéz, szinte biztos, hogy nem
tudjuk mit kell tenni. A bizonytalanság a terv része kell legyen.

Elkezdeni viszont valószínűleg nem bonyolult. Megkérdezhetjük magunkat,
hogy ha lenne időgépünk, visszamennénk pár évvel, hogy már akkor
elkezdjük? Ha már pár évvel ezelőtt elkezdtük volna, örülnénk annak,
hogy mostanra már legalább van némi ismeretünk a helyzetről. A mostban a
jövőbeli önmagunknak tehetjük meg ezt a szívességet. Elkezdhetjük most,
és elképzelhetjük, hogy pár év múlva visszanézünk, és megköszönjük
magunknak, hogy elkezdtük eloszlatni a ködöt.

Amikor nincs tisztán elhatározott szándékunk, csak úgy sodródunk, és
\emph{nem kifejezetten zavar} minket, hogy itt vagyunk, de az elme
szürke és élettelen, szinte próbál elbújni és láthatatlanná válni. Az
eredmény, hogy valóban szürkévé és láthatatlanná válunk így. Semmi rossz
nem történik, de nincs semmi világosság sem abban, hogy ott vagyunk.

Visszanézve a jelenre a jövőbeli önmagunk szemével, visszajönnénk, mert
itt \emph{akarunk} lenni? Lehetünk cinikusak és gondolhatunk a
legrosszabbra, de meglepő módon, a válasz gyakran nem a helyzetet kezdi
el boncolgatni, hanem mint amikor új helyre utazunk, hálásak vagyunk,
hogy ilyen szerencsések vagyunk, hogy itt lehetünk ahol vagyunk. Van
amit még később meg kell tenni, de már azért is köszönetet tudunk
mondani, amit eddig megtapasztalhattunk.

Nem állunk meg elég gyakran, hogy észrevegyük mikor boldogok és
nyugodtak vagyunk. Amikor az elme tiszta és csendes, természetes módon
hálás azért ami itt van, és képes köszönetet mondani az áldásokért amit
életünkben kaptunk. A jelen jó, és akárhogy is alakul az életünk
hátralevő részében, meg tudjuk azt köszönni.

Nem létrehozunk valamit, hanem tiszta szándékkal felismerjük azt ami itt
van. Nem erő vagy képesség kérdése, ezek időhöz és körülményhez
kötöttek. Az elhatározás, a befelé irányuló flismerő figyelem nem egy
adott körülményhez kötött. Az eredménye a helyes szemlélet, amiben
látjuk a dolgok megfelelő helyét, és mit kell azokkal tenni -- vagy csak
megállni, figyelni és lélegezni.
