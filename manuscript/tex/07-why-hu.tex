\chapter{Miért}

\section{Kétség és Hit}

Az érdeklődést a meditáció gyakorlására gyakran a szenvedés érzése
indítja el, érezzük annak szükségét, hogy megbirkózzunk egy felkavaró
vagy fájdalmas tapasztalattal. Tudjuk, hogy `valami nincs rendjén', és
nem tudjuk lerázni az érzést, vagy lehet a veszteség fájdalma, ami olyan
zavar érzését kelti, amiben látszólag semminek sincs értelme -- az ilyen
érzések egyre csak visszatérnek, nem engedik magukat figyelmen kívül
hagyni. Válaszok helyett, csak a gondolataink járnak körbe-körbe: `Miért
kell ennek így lennie? Mit kellene tegyek, és miért? Mi értelme ennek?'

Nem is foglalkozunk sokat az éberséggel és helyes nézettel, amíg az élet
rá nem döbbent minket, hogy mennyi zavart és fájdalmat okoz egy olyan
nézet, ami nem egyezik azzal, ahogy a dolgok vannak. Még az ilyen zavart
állapotban is, pusztán elismerni magunknak a belső káosz állapotát is
már elkezd irányt és rendet teremteni.

Ahhoz hasonlít ez, mintha olyan úton vezetnénk, ami tele van szórva
szeméttel. Lelassítani és körül nézni máris sokkal jobb, mint vaknak
lenni a veszélyes hulladékra. Tele van a fejünk gondolatokkal, de
közöttük kevés mutat megbízható irányba, így jobb, ha megvizsgáljuk
őket. Korábban volt egy elképzelésünk, hogy a dolgok hogy állnak a
világunkban, de ezek megváltoztak olyan módon, amire nem számítottunk.
Nem a dolgok hibája, nem a mi hibánk, de a váratlan változás megzavaró,
és nézetünket igazítanunk kell.

Az állandótlanság tapasztalata kirántja a lábunk alól a szőnyeget, de
egyúttal át is alakítja az értékeinket, azokat a jellemzőket, amiket
keresünk és értékesnek tartunk a tapasztalatainkban. Ha nem értjük a
változást, az zavart okoz, amit kétség követ. Talán tudjuk, mit
\emph{kellene tegyünk}, de a kétség és jelentés nélküliség érzetében
elakadunk, és el sem tudjuk kezdeni.

Miért kelsz fel reggel, hogy tegyél bármit is? Miért számít egyáltalán?
Ha folytatom a `miért' kérdéseket, és így beleások a felépített énem
rétegeibe, az első réteg a megszokott rutint tárja fel, `mert ezt
csináltam tegnap is.' Ez alatt, a válaszokat olyan történek formálják,
amiket én mondok magamnak arról a világról, amiben élek. Ez alatt,
valamilyen érvelés, filozófia és absztrakt ötletek. Ez alatt, kezdek
kétségbeesetten kapaszkodni valamibe, és úgy küszködök, hogy az
elképzeléseimet személyes emlékekkel és tapasztalatokkal védem (`mert
amikor így megy így voltam\ldots{}'), vagy híres személyekre hivatkozok
(`mert ez és az a tanító az mondta\ldots{}'). Ez alatt, fel kell adjam
és bevallanom, hogy a dolog hit és személyes meggyőződés kérdése.
Egyszerűen az, amit eldöntöttem ott és akkor. A végén ott állok, hogy
\emph{nem tudom, de úgy hiszem}, hogy azt tenni értelmes dolog.

A hit nem egy rögzített jellemző az elmében, megvan a kapacitásunk, hogy
megválasszuk a hitelt érdemlő állításokat amiket úgy látunk, hogy
nagyobb megismerés és boldogság felé irányítanak minket, és támogassuk
vagy elhagyjuk azt a hitet azzal, hogy gyakorlatban alkalmazzuk és
figyelünk az eredményekre.

Ellenőrizhetem, felülvizsgálhatom és frissíthetem \emph{mit hiszek}
arról, hogy minek van értelme, de amíg a tapasztalatom nem igazolja azt,
az érvelésemet a hitnek kell alátámasztania, más különben nem fogok
erőfeszítést tenni semmilyen irányba, és az életemet vak szokások és
külső nyomások fogja uralni. A hit az elhatározás és energia erényeinek
üzemanyaga. Később, a hitet megerősíti az, hogy magunkon érezzük a
gyakorlás eredményeit, de üzemanyag nélkül, az autónk el sem indul.

A hit okot teremt a tettre. Hit nélkül, nem teszem meg az adott dolgot.
A buddhista látásmódban két alapvető hittétel van:

\begin{enumerate}
\item
  Egy jelenség megtörténik, ha az elégséges feltételek megvannak hozzá,
  és nem történik meg, vagy megszűnik, ha az elégséges feltételek
  hiányoznak.
  (\href{https://www.accesstoinsight.org/tipitaka/sn/sn12/sn12.061.than.html}{SN
  12.61})
\item
  A Buddha teljesen megértette az igazságot arról, ahogy a dolgok
  vannak, és így megszabadította magát a mohóságtól, gyűlölettől és
  zavarodottságtól, és így kiváló tanítója a gyakorlás útjának.
\end{enumerate}

A Buddha arra adott utasítást, hogy mindenkinek magának kell
rákérdeznie, töprengenie, és megvizsgálnia azt, hogyan vannak valójában
a dolgok, hogy megértse azt. De mégis, hogyan fogjunk hozzá? A tanítóban
való hit nélkül, el vagyunk veszve a személyes véleményeink
kuszaságában, és nem valószínű, hogy hallgatni fogunk másra és tanulni
valamit, ami számunkra új. A hagyomány erre a kapcsolatra emlékeztet
minket, amikor egy Dhamma beszéd előtt, a \emph{namo tassa} sorait
kántáljuk háromszor.

\begin{quote}
\emph{Namo tassa bhagavato arahato sammā-sambuddhassa}

Tisztelet a méltóságos Magasztosnak, a saját erejéből tökéletesen
megvilágosodottnak.
\end{quote}

A Buddha ahhoz hasonlította a kétséget, mintha egy sivatagban
bolyonganánk víz nélkül.\footnote{\href{https://suttacentral.net/dn2/en/sujato}{DN
  2}} Minden más másodlagos, csak arra tudunk gondolni, hogyan találjunk
vizet és jussunk ki a sivatagból. Vagy, hogyan tompítsuk el az elmét és
egyáltalán ne gondoljunk semmire, `álomba ringatni magunkat a
trivialitással, hogy mindent úgy folytathassunk mint eddig.'\footnote{Sören
  Kierkegaard} Nem világos, hogyan fogunk megmenekülni ebből a
helyzetből, de kezdhetjük azzal, hogy elismerjük magunkban a törekvést,
hogy jól legyünk és boldog életet éljünk.

Emberi természetes képességünk, hogy túljussunk a zavaron és hosszú távú
boldogságot fejlesszünk az életünkben. A hosszú távú szemléletnek magába
kell foglalnia a helyzetünk változását, a veszteséget és tragédiát. A
stabil boldogságnak olyan szemléletre kell alapulnia, amely magába építi
az állandótlanságot.

A szuttákban, a kétség szerepel az Öt Akadály\footnote{FIXME definition}
és Tíz Béklyó\footnote{FIXME definition} listáján is. Mint akadály,
megállít minket az elme fejlesztésében, és mint béklyó, arra késztet,
hogy fix bizonyosságokat keressünk, és így még szorosabban köt minket az
elképzeléseinkhez arról, kik vagyunk.

Szeretjük a javaslatot, hogy fejlesszük az elménket, de kezdetben azt
gondoljuk, ez azt jelenti, hogy \emph{megerősítsük} kik és mik vagyunk,
\emph{többet szerezzünk} abból amire szükségünk van, vagy
\emph{megváltoztassunk magunkat} és valami mássá váljunk. `Ki vagyok? Mi
vagyok? Mit kellene tegyek? Ez itt a megfelelő dolog, vagy egy másik?'
Ez a fajta gondolkodás egy csapda, körbe-körbe jár kiút nélkül. Mindezek
a kérdések ahhoz kötődnek, hogy valamilyen ön-azonossághoz
ragaszkodjunk, ami újra kétséget fog magával hozni, és amíg nem vesszük
észre mi történik, benne ragadunk a körforgásban.

Még amikor sikeresek is vagyunk, annak a végén, hogy valamivé váltunk,
az meg fog változni saját természeténél fogva, és az új azonosságot is
hiábavalónak találjuk, üresnek valódi érték nélkül. Ahhoz ragaszkodni
amiről azt gondoljuk, mi vagyunk, félni az elengedéstől: ez az akadály,
ez hozza létre éppen azt a korlátot, aminek frusztráltan neki ütközünk.
A szabadságot az elengedésen keresztül megérteni nem megy nekünk
egykönnyen.

\section{Helyes Nézet}

Térjünk vissza a légzéshez és folytassuk a meditáció gyakorlását. Kezdd
a gyakorlat elejét az alapokkal, egyszerű lépésekkel amik vissza terelik
a figyelmet a jól ismert keretbe: be- és kilégzés, a test és annak
érzései. Figyeld amit tapasztalsz folyamatként, ami állandó változáson
keresztül egyik érzésből a másikba alakul át. Minden meditáció egy új
kezdet, nem menthetjük el az előző alkalom eredményeit, hogy most
betöltsük azt. Ha úgy kezdjük, hogy azt gondoljuk mi már ezt tudjuk,
mondjuk, ha már évek óta gyakoroljuk ezt, ez egy zárt hozzáálláshoz
vezet, ami blokkolja a korábbi megértésünket is, mert az csak úgy
értékes, ha a jelenre vonatkozóan alkalmazzuk. A változó jelen a
megértést megújítja és frissen tartja.

`Ennek a gondolatnak, ennek az érzésnek volt kezdete, most változik, meg
fog szűnni és vége lesz. Meg tudom várni és észre venni azt az
elmúlást?' A közvetlen tapasztalatot így szemlélve, az elme feladja a
vágyat és félelmet az adott állapotokkal kapcsolatban, és megérti őket
mint természetes folyamatok részeit. Nem azt bizonygatjuk magunkban mit
gondolunk az elméről, hanem mintha egy lépést hátralépve szemlélnénk,
éberen tapasztaljuk azt, ahogy van.

Ez a szemlélődés vissza állítja a helyes nézetet, mintha egy fordítva, a
fején álló virág vázát valaki újra egyenesen felállítana, és mikor
ránézünk értjük, hogy a vázának melyik az alja és a teteje. Szomjasan
kívánjuk és ragaszkodunk olyan tapasztalatokhoz, amik mindig változni
fognak; nem hangzik ez feszültségnek és szenvedésnek? Szerencsére a hiba
elkerülhető.

A helyes nézet megtalálja a teret és szabadságot az élet korlátai és
nyomásai körül. Eleinte talán nem látunk túl sok szabad teret, de a
lényeges dolgokat megvizsgálva észrevehetjük, hogy nincs szükségünk
mindenre amire gondolni tudunk. Megkérdezhetjük, `Megvan minden, amire
szükségem van az élethez erre az egy napra?' Sorra vehetjük mit
használunk a közvetlen környezetünkben -- ruha, étel, szállás,
gyógyszerek. Egyszer mi kapjuk valakitől, vagy engedik, hogy használjuk,
máskor mi adjuk másoknak. `Tudom, mit elég megtennem a mai nap?'
Alighanem már ismerem ezeket a tényeket, de úgy érzem visszatér a
nyugalom, mikor újra felidézem őket.

Ezzel az egyszerű ténnyel, hogy megvan amire szükségünk van, hogy jól
éljük ezt a napot, a hozzáállásunk abban fejezi ki magát, hogy
megnyugvást és hálát érzünk az életért. Nem kell kérned ezt, és nem
tudod akarattal létrehozni. Teret kell adni neki a szemléletünkben, és
magától megjelenik.

Hova ez a nagy sietség? Emlékezhetünk Thoreau szavaira \emph{Walden} c.
könyvéből, ``Rossz, ha valakinek déli hajcsára van, még rosszabb, ha az
északi államokból való; de a legrosszabb, ha magad vagy a magad
hajcsára!'' Egy egyszerű gyakorlat, hogy megállunk két percre, nem
keresni szórakoztatást és figyelemelterelést, egyszerűen semmit nem
tenni két percig. Figyelheted a lélegzetet, de ez is választás kérdése.
Próbáld meg néhányszor nap közben, figyeld meg tesz-e különbséget.

A probléma nem az, hogy nincs nem tudunk eleget. A könyvespolcok
túlcsordulnak a jó tanáccsal arról, `hogyan legyünk boldogak', ha csak
ez kellene, akkor hol a hiba? Ha csak a jó tanácson múlna, már
mindannyian rég megvilágosodtunk volna. Halljuk és olvasunk arról, hogy
mi minden jó dolgot kellene tennünk, milyenféle embernek kellene
lennünk: az egyik könyv szerint legyünk kemények és félelem nélküliek,
miközben a másik szerint univerzális együttérzésre van szükségünk. Ez
külön fajta speciális szenvedés végigolvasni az egészet.

Vagy talán a \emph{Nibbánára} van szükségünk? Ez a megfelelő dolog? A
szó jelentése \emph{elhűlt, hűvös}, gondolhatunk egy tűzre, ami kialszik
és elhűl. A szomjas vágy, hogy `megszerezzük', csak több tüzelőanyagot
jelentene a létesülés hőségéhez és tovább égéséhez. De a \emph{Nibbána}
a létesülésben égés kialvásának hűvössége, tehát ilyen nem-létesüléssé
kellene váljunk? A gondolkodó elme erre az mondja, `\emph{Mi van?!}' És
ez nem is rossz válasz: a Buddha tanítása arra mutat rá, hogy a
gondolkodás és létesülés nem elégséges ehhez. Egy újabb állapot vagy
gondolat, mikor magunkat látjuk benne, olyan korlátozó lesz mint a
korábbi. Nem abban áll a szabadságunk, hogy a megfelelő dologgá válunk,
hanem a felismerésben, hogy fel tudjuk adni a kényszert, hogy a folyton
valamivé válnunk kelljen.

\section{Új Szemmel}

\begin{verbatim}
We can turn a compulsive tendency into right view by asking, 'How can I understand this experience?'
This question directs us to the noble attitude to suffering described in the First Noble Truth: 'Suffering should be understood.'
The purpose is not to figure out who to blame or to imagine personal stories in our head.
Right view understands what is conditioned phenomena, what is impermanence, suffering and not-self.
It is best to put aside the opinions which present themselves as answers,
and keep returning to the open attitude of knowing the present.
\end{verbatim}

Egy kényszerű hajlamot helyes nézetté változtathatunk át, ha
megkérdezzük, `Hogyan tudom megérteni ezt a tapasztalatot?' Ez a kérdés
a nemes hozzáállás felé irányít minket, amit az Első Nemes Igazság
tartalmaz: `A szenvedést meg kell érteni.' A cél nem az, hogy rájöjjünk
kit kell hibáztatnunk, vagy, hogy személyes történeteket képzeljünk el a
fejünkben. A helyes nézet azt érti, hogy mik a feltételekhez kötődő
jelenségek, mi az állandótlanság, szenvedés és éntelenség. Legjobb félre
tenni a véleményeket, melyek válaszként mutatkoznak, és folyton
visszatérni ahhoz a nyitott hozzáálláshoz, ami ismeri a jelen
pillanatot.

\begin{verbatim}
Both joy and sorrow are natural processes,
but if we don't understand them, we see one as a reward and the other as punishment.
Looking at it this way, is seem that life is never fair and it always seems to be out of our control.
\end{verbatim}

Az öröm és a bánat mind természetes folyamatos, de ha nem értjük őket,
az egyiket jutalomnak tekintjük, a másikat pedig büntetésnek. Így nézve,
úgy tűnik, az élet sosem igazságos, és mindig úgy tűnik, mint ami
irányításunkon kívül esik.

\begin{verbatim}
We arrive at a turning point when we are not sure about our opinions and we stop to investigate the experience itself.
To open up our attitude for contemplation, we can at least imagine the possibility
that there is something here we can learn.
Consider how narrow our perspective is when we start with the thought, '/I've seen this, I know this/'.
Perhaps this is true, and I have seen and recognized it before.
But I notice that when I try to use that intellectual knowledge to solve a problem,
my attention revolves around memories, thoughts and opinions.
While I am caught up in the past, the present experience escapes my attention.
\end{verbatim}

Egy fordulóponthoz érkezünk, amikor nem vagyunk biztosak a
véleményeinkben, és megállunk megvizsgálni magát a tapasztalatot. Ahhoz,
hogy megnyissuk a hozzáállásunkat a vizsgálódáshoz, legalább el kell
tudjuk képzelni a lehetőséget, hogy van itt valami amit meg tudunk
tanulni. Vedd figyelembe, milyen szűk a szemléletünk, amikor azzal a
gondolattal kezdünk, hogy `\emph{Ezt már láttam, én ezt ismerem.}'
Lehet, hogy ez igaz, és már láttam és felismertem korábban. De azt
veszem észre, hogy amikor ezt az intellektuális információt próbálom
használni egy probléma megoldásához, a figyelmem emlékek, gondolatok és
vélemények körül forog. Amíg magával ragad a múlt, a jelen tapasztalat
elkerüli a figyelmem.

\begin{verbatim}
The instruction of the Buddha is to carefully establish the intention to meditate,
and to put aside the matters of the world.
We are to watch experiences directly as they are,
not thinking about how we think they should be.
\end{verbatim}

A Buddha utasítása, hogy óvatosan alapozzuk meg a szándékunkat a
meditációra, és tegyük félre a világ ügyeit. Figyelnünk kell a
tapasztaltunkat, ahogy azok vannak, nem arról gondolkodni, hogyan
kellene legyenek.

\begin{quote}
Úgy időzik, hogy a testet, {[}az érzéseket, a tudatot, a dhammákban{]} a
keletkezés\ldots{} az elmúlás\ldots{} vagy a keletkezés és az elmúlás
természetét szemléli. Megalapozódik benne az éberség: „van test,
{[}vannak érzések, van tudat, vannak dhammák{]}``, oly mértékben, amely
a puszta tudáshoz és a folytonos éberséghez szükséges. Szabadon időzik,
semmihez sem kötődve a világon.\footnote{MN 10}
\end{quote}

\begin{verbatim}
The present experience of the six senses is continuously changing,
thoughts and opinions don't become 'our knowledge',
but we may understand the process of their arising and ceasing.
'/What/ is it that I am doing? /How/ am I doing it?'
Letting go of our fixed positions shows the way forward;
we discover it by seeing with new eyes.[fn:: "The real voyage of discovery consists not in seeking new landscapes, but in having new eyes." (Marcel Proust)]
Life may still not be fair and is not entirely under our control,
but now we are familiar with a practice which makes the difference between
having a mental breakdown and facing the facts with understanding.
\end{verbatim}

A hat érzék jelenbeli tapasztalata folyamatosan változik, a gondolatok
és vélemények nem válnak a `saját tudásunkká', de megérthetjük a
megjelenésük és megszűnésük folyamatát. `\emph{Mi} az, amit éppen
teszek? \emph{Hogyan} teszem azt?' Az előre vezető utat az mutatja, hogy
elengedjük a merev álláspontjainkat; úgy fedezzük fel, hogy új szemmel
látunk.\footnote{``Az igazi felfedezőúthoz nem más tájakra van szükség,
  hanem új szemre.'' (Marcel Proust)} Az élet talán továbbra sem
igazságos és nincs egészen az irányításunk alatt, de most már ismerünk
egy gyakorlást, ami a különbséget jelenti aközött, hogy teljesen
kiborulunk vagy megértéssel szembenézünk a tényekkel.

\begin{verbatim}
The fundamental principle is that watching the mind develops the mind.
Conscious awareness stops the accumulated tendencies which have been unseen but creating compulsions under the surface.
Slowly we learn to trust our capacity to practise this wakefulness in the present,
which understands that change is an essential part of being alive.
The word 'Buddha' means 'one who knows, one who is awake'.
The source of contentment in activity is that we trust and turn to the wakeful awareness,
which recognizes the truth, and through the virtues of every day, lives in it.
\end{verbatim}

Az alapelv, hogy figyelni az elmét, fejleszti az elmét. Az éber
tudatosság megállítja a felgyülemlett hajlamokat, amiket eddig nem
láttunk, de a felszín alatt kényszereket hoztak létre. Lassan
megtanulunk bízni a képességünkben, hogy a jelenben gyakoroljuk ezt az
éberséget, ami megérti, hogy a változás lényegi része annak, hogy
életben vagyunk. A `Buddha' szó azt jelenti, `aki megismer, aki éber'. A
tevékenységekben található megelégedettség forrása az, hogy bizalommal
fordulunk az éber figyelemhez, ami felismeri az igazságot, és a
mindennapi erényeken keresztül, abban él.
